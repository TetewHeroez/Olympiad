\documentclass[a4paper, 11pt]{article}

\usepackage{geometry}
\usepackage{tikz}
    \usetikzlibrary{patterns,decorations.pathreplacing,snakes,arrows.meta}
    \tikzset{box/.style={draw, thick, minimum width=1cm, minimum height=1cm}}
\usepackage{bigints}
\usepackage{tabularx}
\usepackage{fancyhdr}
\usepackage{amsfonts}
\usepackage{graphicx}
\usepackage{fontenc}
\usepackage{amssymb}
\usepackage{amsmath}
\usepackage{amsthm}
\usepackage{footmisc} 
\usepackage[colorlinks=true, linkcolor=red]{hyperref}
\usepackage{multicol}
\usepackage{enumitem}
\usepackage{cancel}
%\usepackage{wrapfig}
\usepackage{longtable}
\usepackage{array}
\usepackage{bm}
\usepackage{color}
\usepackage{textcomp}
\usepackage{xcolor}
\usepackage{mdframed}
\definecolor{darkcyan}{HTML}{0091A4}  
\definecolor{brightcyan}{HTML}{dcf0f2}
\newmdenv[  
topline=false,  
rightline=false,  
bottomline=false,  
leftline=true,  
linecolor=darkcyan,  
linewidth=3pt,  
backgroundcolor=brightcyan,  
frametitle={\textit{Solusi}:}
]{solution}  
\setlength{\multicolsep}{5.0pt plus 2.0pt minus 1.5pt}% 50% of original values
\geometry{a4paper, portrait, top=2.5cm, left=2.5cm, right=2.5cm, bottom=2.5cm}

\newcommand{\R}{\mathbb{R}}
\newcommand{\Z}{\mathbb{Z}}
\newcommand{\C}{\mathbb{C}}
\newcommand{\N}{\mathbb{N}}
\newcommand{\Q}{\mathbb{Q}}
\newcommand{\F}{\mathbb{F}}

\renewcommand{\baselinestretch}{1.2}

\fancyfoot[L]{\textit{5002221132}}
\fancyfoot[R]{\textit{Tetew}}
\renewcommand{\headrulewidth}{0pt}
\renewcommand{\footrulewidth}{2pt}

\pagenumbering{gobble}
\begin{document}
\begin{center}
  \textbf{SELEKSI SESI PERTAMA BIDANG MATEMATIKA} \\[6pt]

  \begin{tabular}{ l l }
    Hari, Tanggal           & : Sabtu, 16 Agustus 2025 \\
    Durasi pengerjaan       & : 90 menit               \\
    Waktu pengerjaan        & : Pukul 08.00--09.30 WIB \\
    Batas waktu pengumpulan & : Pukul 09.40 WIB        \\
    Total poin              & : 30                     \\
  \end{tabular}
\end{center}
\noindent
\textbf{Naskah soal ini terdiri atas dua halaman.}

\begin{enumerate}
  \item Diberikan ruang vektor $V$ atas lapangan $\mathbb{R}$ dengan $\dim(V)=6$. Diketahui $U_1$ dan $U_2$ merupakan subruang dari $V$ dengan $U_1 \subseteq U_2$ dan $\dim(U_1)=2$ serta $\dim(U_2)=4$. Jika
        \[
          E = \{ T : V \to V \mid T \text{ transformasi linear},\; T(U_1) \subseteq U_1, \; T(U_2) \subseteq U_2 \},
        \]
        maka $\dim(E) = \dots$.

  \item Bilangan tiga digit $abc$ dikatakan ``jaya'' jika sistem persamaan linear
        \[
          \begin{cases}
            ax + by + cz = a, \\
            cx + ay + bz = b, \\
            bx + cy + az = c
          \end{cases}
        \]
        mempunyai lebih dari dua solusi. Banyaknya bilangan jaya adalah \dots

  \item Jika bilangan kompleks $z$ memenuhi $z = (1+i)^i$, maka nilai yang mungkin dari $\arg(z)$ adalah \dots

  \item Jika $a$ dan $b$ adalah bilangan kompleks dengan sifat $|a| \leq |b|$, maka nilai terkecil dari $|z|, z \in \mathbb{C}$, yang memenuhi persamaan
        \[
          |z-a| + |z+a| = 2|b|
        \]
        adalah \dots

  \item Nilai dari
        \[
          \lim_{n \to \infty} \frac{1}{n^{26}} \sum_{k=1}^{n} (k+25)^{25}
        \]
        adalah \dots

  \item Diberikan fungsi terdiferensial $f : \mathbb{R} \to \mathbb{R}$ yang memenuhi $f'(0) = 1$ dan
        \[
          f(x+y) = \frac{f(x)}{e^y} + \frac{f(y)}{e^x}, \quad \forall x,y \in \mathbb{R}.
        \]
        Nilai terbesar yang mungkin dari $f(1)$ adalah \dots
  \item Sebuah permainan terdiri dari dua jenis langkah: maju dan mundur. Permainan akan selesai apabila salah satu jenis langkah dilakukan sebanyak 5 kali, tanpa memperhatikan urutannya. Banyaknya kemungkinan kombinasi langkah pada suatu permainan yang selesai tepat setelah langkah terakhir adalah ....

  \item Suatu bilangan bulat positif dikatakan bebas kuadrat (\textit{squarefree}) jika bilangan bulat itu tidak habis dibagi oleh bilangan kuadrat yang lebih besar dari 1. Contoh bilangan bebas kuadrat adalah 1 dan 21. Banyaknya bilangan bebas kuadrat yang kurang dari 250 adalah ....

  \item Banyaknya anggota dari grup faktor $\mathbb{Q}/\mathbb{Z}$ yang berorde bilangan prima yang kurang dari 50 adalah ....

  \item Banyaknya pembagi nol berderajat satu di gelanggang suku banyak $ \mathbb{Z}_{10}[x] $ adalah ....
\end{enumerate}
\newpage
\pagestyle{fancy}
\begin{center}
  \textbf{\underline{SOLUSI}}
  \begin{enumerate}
    \item Misalkan $B_1 = \{u_1, u_2\}$ adalah basis dari $U_1$. Karena $U_1 \subseteq U_2$ dan $\dim(U_2) = 4$, maka kita dapat memperluas $B_1$ menjadi basis $B_2 = \{u_1, u_2, u_3, u_4\}$ dari $U_2$. Selanjutnya, karena $U_2 \subseteq V$ dan $\dim(V) = 6$, kita dapat memperluasnya kembali sehingga diperoleh $B_3=\{u_1, u_2, u_3, u_4, v_1, v_2\}$ adalah basis dari $V$.

          Misalkan $T \in E$. Karena $T(U_1) \subseteq U_1$, maka
          \[
            T(u_1) = a_{11}u_1 + a_{21}u_2, \quad T(u_2) = a_{12}u_1 + a_{22}u_2.
          \]
          Selanjutnya, karena $T(U_2) \subseteq U_2$, maka
          \[
            T(u_3) = a_{13}u_1 + a_{23}u_2 + a_{33}u_3 + a_{43}u_4, \quad T(u_4) = a_{14}u_1 + a_{24}u_2 + a_{34}u_3 + a_{44}u_4.
          \]
          Terakhir, karena $T$ adalah transformasi linear dari $V$ ke $V$, maka
          \[
            T(v_1) = b_{11}u_1 + b_{21}u_2 + b_{31}u_3 + b_{41}u_4 + b_{51}v_1 + b_{61}v_2,
          \]
          \[
            T(v_2) = b_{12}u_1 + b_{22}u_2 + b_{32}u_3 + b_{42}u_4 + b_{52}v_1 + b_{62}v_2.
          \]

          Dengan demikian, transformasi linear $T$ ditentukan oleh 24 bilangan real bebas, yaitu $a_{11}, a_{21}, a_{12}, a_{22}, a_{13}, a_{23}, a_{33}, a_{43}, a_{14}, a_{24}, a_{34}, a_{44}, b_{11}, b_{21}, b_{31}, b_{41}, b_{51}, b_{61}, b_{12}, b_{22}, b_{32}, b_{42}, b_{52}, b_{62}$.

          Jadi, $\dim(E) = 24$.

    \item
  \end{enumerate}
\end{center}
\end{document}