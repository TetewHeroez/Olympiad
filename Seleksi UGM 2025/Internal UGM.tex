\documentclass[a4paper, 12pt]{article}

\usepackage{geometry}
\usepackage{tikz}
    \usetikzlibrary{patterns,decorations.pathreplacing,snakes,arrows.meta}
    \tikzset{box/.style={draw, thick, minimum width=1cm, minimum height=1cm}}
\usepackage{bigints}
\usepackage{tabularx}
\usepackage{fancyhdr}
\usepackage{amsfonts}
\usepackage{graphicx}
\usepackage{fontenc}
\usepackage{amssymb}
\usepackage{amsmath}
\usepackage{amsthm}
\usepackage{footmisc} 
\usepackage[colorlinks=true, linkcolor=red]{hyperref}
\usepackage{multicol}
\usepackage{enumitem}
%\usepackage{wrapfig}
\usepackage{longtable}
\usepackage{array}
\usepackage{bm}
\usepackage{color}
\usepackage{textcomp}
\usepackage{xcolor}
\usepackage{mdframed}
\definecolor{darkcyan}{HTML}{0091A4}  
\definecolor{brightcyan}{HTML}{dcf0f2}
\newmdenv[  
topline=false,  
rightline=false,  
bottomline=false,  
leftline=true,  
linecolor=darkcyan,  
linewidth=3pt,  
backgroundcolor=brightcyan,  
frametitle={\textit{Solusi}:}
]{solution}  
\setlength{\multicolsep}{5.0pt plus 2.0pt minus 1.5pt}% 50% of original values
\geometry{a4paper, portrait, top=2.5cm, left=2.5cm, right=2.5cm, bottom=2.5cm}

\newcommand{\R}{\mathbb{R}}
\newcommand{\Z}{\mathbb{Z}}
\newcommand{\C}{\mathbb{C}}
\newcommand{\N}{\mathbb{N}}
\newcommand{\Q}{\mathbb{Q}}
\newcommand{\F}{\mathbb{F}}

\renewcommand{\baselinestretch}{1.2}

\fancyfoot[L]{\textit{5002221132}}
\fancyfoot[R]{\textit{Tetew}}
\renewcommand{\headrulewidth}{0pt}
\renewcommand{\footrulewidth}{2pt}
\pagestyle{fancy}
\pagenumbering{gobble}
\begin{document}
\section*{Isian singkat}
\begin{enumerate}
  \item Diberikan matriks \( A \) berukuran \( 2025 \times 2025 \) dengan
        \[
          \det(A - \lambda I_{2025 \times 2025}) = (\lambda - 2)^k (\lambda - 1)^{2025-k}
        \]
        untuk suatu bilangan asli\footnote{Agak rancu disini karena didefinisikan bilangan asli namun nilai $k=0$ disebutkan di kalimat setelahnya.} \( k \) dengan \( 0 \leq k \leq 2025 \). Jika \( A^2 = A \), maka banyaknya nilai \( k \) yang mungkin adalah \ldots

        \begin{solution}
          Bentuk determinan yang diberikan menunjukkan bahwa nilai eigen 2 muncul sebanyak \( k \) kali, dan nilai eigen 1 muncul sebanyak \( 2025 - k \) kali. Sedangkan diketahui bahwa \( A \) adalah matriks idempotent, sehingga nilai eigen dari \( A \) hanya bisa $\lambda=0$ atau $\lambda=1$.

          Dengan demikian, \( k \) harus sama dengan 0, karena jika \( k > 0 \), maka akan ada nilai eigen 2 yang tidak sesuai dengan sifat idempotent. Oleh karena itu, satu-satunya nilai yang mungkin untuk \( k \) adalah $\boxed{1}$ saja yaitu \( k=0 \).
        \end{solution}

  \item Diberikan ruang vektor \( V \) atas lapangan \( F \), dengan \( \dim(V) = 7 \), serta transformasi linear \( T_1 : V \to V \) dan \( T_2 : V \to V \), dengan \( \dim(\mathrm{Im}(T_1)) = 3 \) dan \( \dim(\mathrm{Im}(T_2)) = 4 \). Jika \( M \) dan \( m \) berturut-turut menyatakan nilai terbesar dan terkecil yang mungkin dari \( \dim(\mathrm{Im}(T_2 \circ T_1)) \), maka nilai \( M + m = \ldots \)

        \begin{solution}
          Dengan menggunakan Teorema Rank-Nullity, kita tau bahwa
          \[\dim(\mathrm{Im}(T_2\circ T_1)) \leq \min\{\dim(\mathrm{Im}(T_1)), \dim(\mathrm{Im}(T_2))\}\]
          Sehingga dengan jelas kita peroleh
          \[\dim(\mathrm{Im}(T_2\circ T_1)) \leq \min\{3,4\} = 3 = M\]
        \end{solution}

  \item Jika setiap \( z \in \mathbb{C} \) yang memenuhi
        \[
          \left| \frac{z+1}{z+4} \right| = 2
        \]
        terletak pada suatu lingkaran, maka radius dan titik pusat lingkaran tersebut berturut-turut adalah \ldots

  \item Bentuk sederhana dari
        \[
          \sum_{n=1}^{\infty} 3^{n-1} \sin^3\left( \frac{x}{3^n} \right)
        \]
        adalah \ldots

  \item Nilai
        \[
          \sup \left\{ \inf \left\{ 5(-1)^n - \left( \frac{m+1}{n} \right)^2 : n \geq m \right\} : m \in \mathbb{N} \right\}
        \]
        adalah \ldots

  \item Diberikan fungsi kontinu \( f : [0,1] \to \mathbb{R} \) dengan \( |f(x)| \leq x \) untuk setiap \( x \in [0,1] \). \\
        Nilai terbesar yang mungkin dari
        \[
          \int_0^1 \left( (f(x))^2 - x^4 f(x) \right) \, dx
        \]
        adalah \ldots

  \item Banyaknya bilangan asli \( x \in \{1,2,3,\ldots,2025\} \) yang bukan kelipatan 2 dan bukan kelipatan 5 adalah \ldots

  \item Tiga siswa $a_{1}$, $a_{2}$, $a_{3}$ dari Sekolah A dan 4 siswa $b_{1}$, $b_{2}$, $b_{3}$, $b_{4}$ dari Sekolah B berkumpul dalam sebuah pertemuan. % 
        Banyaknya cara menyusun ketujuh siswa tersebut dalam satu baris dengan syarat tidak terdapat satu blok yang berisikan semua siswa dari sekolah yang sama adalah \ldots % 
        (Contoh: $b_{3}b_{2}a_{1}a_{3}b_{1}b_{4}a_{2}$ diperbolehkan, tetapi $b_{1}b_{4}a_{1}a_{2}a_{3}b_{3}b_{2}$ dan $a_{1}b_{4}b_{3}b_{1}b_{2}a_{3}a_{2}$ tidak diperbolehkan) % 

  \item Jika $S_{5}$ menyatakan grup semua fungsi bijektif pada $\{1,2,..., 5\}$ terhadap operasi komposisi fungsi, maka banyaknya elemen berorder 2 pada $S_{5}$ adalah \ldots % 

  \item Jika $z$, $p$, dan $m$ berturut-turut menyatakan banyaknya ideal di $\mathbb{Z}_{2025}$, banyaknya ideal prima di $\mathbb{Z}_{2025}$, dan banyaknya ideal maksimal di $\mathbb{Z}_{2025}$, maka nilai $z+p+m$ adalah \ldots % 
\end{enumerate}
\section*{Uraian}
\begin{enumerate}
  \item
        Diberikan ruang vektor $V$ atas lapangan $F$ dengan $\dim(V)=n$. Misalkan $U$ dan $W$ merupakan dua ruang bagian dari $V$ dengan $\dim(U)+\dim(W)=n$. % 
        Buktikan bahwa terdapat transformasi linear $T:V\longrightarrow V$ yang memenuhi $\ker(T)=U$ dan $\mathrm{Im}(T)=W.$ % 

  \item
        Diketahui $A_{1},A_{2},...,A_{n}$ merupakan titik-titik sudut sebuah poligon $n$ sisi beraturan yang termuat pada sebuah lingkaran dengan radius $r$ dan titik pusat $O(0,0)$. Jika $P$ merupakan titik di luar lingkaran yang terletak pada garis perpanjangan $OA_{1}$, buktikan bahwa % 
        \[
          \prod_{k=1}^{n}|PA_{k}|=|OP|^{n}-r^{n},
        \] % 
        dengan $|AB|$ menyatakan panjang ruang garis yang menghubungkan titik A dan B. % 

  \item
        Diberikan fungsi kontinu $f:[a,b]\rightarrow\mathbb{R}$ yang terdiferensial pada $(a, b)$ dengan $f(a)=f(b)$. % 
        Tunjukkan bahwa untuk setiap bilangan asli $n$, terdapat $n$ bilangan real berbeda, % 
        $c_{1},c_{2},...,c_{n}\in(a,b)$, yang memenuhi % 
        \[
          f^{\prime}(c_{1})+f^{\prime}(c_{2})+\cdot\cdot\cdot+f^{\prime}(c_{n})=0.
        \] % 

  \item
        Untuk suatu bilangan bulat tak negatif $n$ dan $m$, misalkan $D(m,n)$ menyatakan banyaknya solusi persamaan % 
        \[
          x_{1}+x_{2}+\cdot\cdot\cdot+x_{m}=n
        \] % 
        dengan $x_{i}\in\mathbb{N}$, untuk setiap $i\in\{1,2,...,m\}$ dan $x_{1}<x_{2}<\ldots<x_{m}$. Tunjukkan bahwa % 
        \[
          D(m,n)=D(m,n-m)+D(m-1,n-m).
        \] % 

  \item
        \begin{enumerate}
          \item Diketahui $A$, $B$, $C$ merupakan subgrup dari sebuah grup $G$. Jika $A\subseteq B$, $A\cap C=B\cap C$, dan $AC=BC$, buktikan bahwa $A=B$. % 
          \item Carilah contoh grup $G$ dan subgrup $A$, $B$, $C$ dari $G$ yang memenuhi $A\cap C=B\cap C$, dan $AC=BC$ tetapi $A\ne B$. % 
        \end{enumerate}
        Catatan: Jika $P$ dan $Q$ adalah subgrup dari $(G, *)$, maka $PQ$ didefinisikan sebagai % 
        \[
          PQ=\{p*q:p\in P,q\in Q\}.
        \] % 
  \item
\end{enumerate}
\end{document}