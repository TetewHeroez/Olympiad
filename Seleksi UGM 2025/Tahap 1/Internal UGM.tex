\documentclass[a4paper, 12pt]{article}

\usepackage{geometry}
\usepackage{tikz}
    \usetikzlibrary{patterns,decorations.pathreplacing,snakes,arrows.meta}
    \tikzset{box/.style={draw, thick, minimum width=1cm, minimum height=1cm}}
\usepackage{bigints}
\usepackage{tabularx}
\usepackage{fancyhdr}
\usepackage{amsfonts}
\usepackage{graphicx}
\usepackage{fontenc}
\usepackage{amssymb}
\usepackage{amsmath}
\usepackage{amsthm}
\usepackage{footmisc} 
\usepackage[colorlinks=true, linkcolor=red]{hyperref}
\usepackage{multicol}
\usepackage{enumitem}
\usepackage{cancel}
%\usepackage{wrapfig}
\usepackage{longtable}
\usepackage{array}
\usepackage{bm}
\usepackage{color}
\usepackage{textcomp}
\usepackage{xcolor}
\usepackage{mdframed}
\definecolor{darkcyan}{HTML}{0091A4}  
\definecolor{brightcyan}{HTML}{dcf0f2}
\newmdenv[  
topline=false,  
rightline=false,  
bottomline=false,  
leftline=true,  
linecolor=darkcyan,  
linewidth=3pt,  
backgroundcolor=brightcyan,  
frametitle={\textit{Solusi}:}
]{solution}  
\setlength{\multicolsep}{5.0pt plus 2.0pt minus 1.5pt}% 50% of original values
\geometry{a4paper, portrait, top=2.5cm, left=2.5cm, right=2.5cm, bottom=2.5cm}

\newcommand{\R}{\mathbb{R}}
\newcommand{\Z}{\mathbb{Z}}
\newcommand{\C}{\mathbb{C}}
\newcommand{\N}{\mathbb{N}}
\newcommand{\Q}{\mathbb{Q}}
\newcommand{\F}{\mathbb{F}}

\renewcommand{\baselinestretch}{1.2}

\fancyfoot[L]{\textit{5002221132}}
\fancyfoot[R]{\textit{Tetew}}
\renewcommand{\headrulewidth}{0pt}
\renewcommand{\footrulewidth}{2pt}
\pagestyle{fancy}
\pagenumbering{gobble}
\begin{document}
\section*{Isian singkat}
\begin{enumerate}
  \item Diberikan matriks \( A \) berukuran \( 2025 \times 2025 \) dengan
        \[
          \det(A - \lambda I_{2025 \times 2025}) = (\lambda - 2)^k (\lambda - 1)^{2025-k}
        \]
        untuk suatu bilangan asli\footnote{Agak rancu disini karena didefinisikan bilangan asli namun nilai $k=0$ disebutkan di kalimat setelahnya.} \( k \) dengan \( 0 \leq k \leq 2025 \). Jika \( A^2 = A \), maka banyaknya nilai \( k \) yang mungkin adalah \ldots

        \begin{solution}
          Bentuk determinan yang diberikan menunjukkan bahwa nilai eigen 2 muncul sebanyak \( k \) kali, dan nilai eigen 1 muncul sebanyak \( 2025 - k \) kali. Sedangkan diketahui bahwa \( A \) adalah matriks idempotent, sehingga nilai eigen dari \( A \) hanya bisa $\lambda=0$ atau $\lambda=1$.

          Dengan demikian, \( k \) harus sama dengan 0, karena jika \( k > 0 \), maka akan ada nilai eigen 2 yang tidak sesuai dengan sifat idempotent. Oleh karena itu, satu-satunya nilai yang mungkin untuk \( k \) adalah $\boxed{1}$ saja yaitu \( k=0 \).
        \end{solution}

  \item Diberikan ruang vektor \( V \) atas lapangan \( F \), dengan \( \dim(V) = 7 \), serta transformasi linear \( T_1 : V \to V \) dan \( T_2 : V \to V \), dengan \( \dim(\mathrm{Im}(T_1)) = 3 \) dan \( \dim(\mathrm{Im}(T_2)) = 4 \). Jika \( M \) dan \( m \) berturut-turut menyatakan nilai terbesar dan terkecil yang mungkin dari \( \dim(\mathrm{Im}(T_2 \circ T_1)) \), maka nilai \( M + m = \ldots \)

        \begin{solution}
          Misalkan $S=T_2\circ T_1$ dan dengan menggunakan Teorema Rank-Nullity, kita tau bahwa
          \[\dim(\mathrm{Im}(S)) \leq \min\{\dim(\mathrm{Im}(T_1)), \dim(\mathrm{Im}(T_2))\}\]
          Sehingga dengan jelas kita peroleh
          \[\dim(\mathrm{Im}(S)) \leq \min\{3,4\} = 3 = M\]
          Secara intuitif dapat kita bayangkan bahwa beberapa elemen dari \( \mathrm{Im}(T_1) \) mungkin saja terpenuhi untuk menjadi elemen dari \( \ker(T_2) \), sehingga didapatkan sebuah rumus sebagai berikut
          \[\dim(\ker(S))=\dim(\ker(T_1)) + \dim(\ker(T_2)\cap\mathrm{Im}(T_1))\]
          $\dim(\mathrm{Im}(S))$ diberikan oleh rumus
          \[\dim(\mathrm{Im}(S)) = \dim(V) - \dim(\ker(S))= 7 - \dim(\ker(S))\]
          yang artinya $\dim(\mathrm{Im}(S))$ minimum ketika $\dim(\ker(S))$ maksimum atau lebih lanjutnya $\dim(\ker(T_2)\cap\mathrm{Im}(T_1))$ harus maksimum.

          Disisi lain kita tahu bahwa $\dim(\ker(T_2))=\dim(V)-\dim(\mathrm{Im}(T_2))=7-4=3$. Dengan memilih $\ker(T_2)=\mathrm{Im}(T_1)$ maka dapat kita peroleh
          \[\dim(\mathrm{Im}(S)) = 0 = m\]
          Maka, nilai \( M + m = 3 + 0 = \boxed{3} \).
        \end{solution}

  \item Jika setiap \( z \in \mathbb{C} \) yang memenuhi
        \[
          \left| \frac{z+1}{z+4} \right| = 2
        \]
        terletak pada suatu lingkaran, maka radius dan titik pusat lingkaran tersebut berturut-turut adalah \ldots
        \begin{solution}
          Misalkan \( z = x + iy \) dengan \( x, y \in \mathbb{R} \), maka
          \begin{align*}
            \left| \frac{z+1}{z+4} \right| & = 2                        \\
            \left| z + 1 \right|           & = 2 \left| z + 4 \right|   \\
            \left| z + 1 \right|^2         & = 4 \left| z + 4 \right|^2 \\
            (x+1)^2 + y^2                  & = 4((x+4)^2 + y^2)         \\
            x^2 + 2x + 1 + y^2             & = 4x^2 + 32x + 64 + 4y^2   \\
            3x^2 + 3y^2 + 30x + 63         & = 0                        \\
            x^2 + y^2 + 10x + 21           & = 0                        \\
            (x + 5)^2 + y^2                & = 4.
          \end{align*}
          Dengan demikian, radius lingkaran tersebut adalah \( 2 \) dan titik pusatnya adalah \( (-5, 0) \) atau $z=-5$.
        \end{solution}

  \item Bentuk sederhana dari
        \[
          \sum_{n=1}^{\infty} 3^{n-1} \sin^3\left( \frac{x}{3^n} \right)
        \]
        adalah \ldots
        \begin{solution}
          Perhatikan bahwa $\sin\left(\theta\right)=\dfrac{e^{i\theta}-e^{-i\theta}}{2i}$, sehingga
          \[
            \sin^3\left( \theta \right) = \left( \frac{e^{i\theta}-e^{-i\theta}}{2i} \right)^3 = \frac{1}{-8i} \left( e^{3i\theta} - 3e^{i\theta} + 3e^{-i\theta} - e^{-3i\theta} \right).
          \]
          Jika dikembalikan ke dalam bentuk sinus, kita dapat tuliskan
          \[
            \sin^3\left( \theta \right) = \frac{1}{4}\left[3 \sin(\theta) - \sin(3\theta)\right].
          \]
          Sehingga setiap suku pada deret tersebut dapat dituliskan sebagai
          \[
            3^{n-1} \sin^3\left( \frac{x}{3^n} \right) = \frac{1}{4}\left[3^n \sin\left( \frac{x}{3^{n}} \right) - 3^{n-1} \sin\left( \frac{x}{3^{n-1}} \right)\right].
          \]
          Perhatikan bahwa deret diatas dapat dijadikan teleskopik seperti berikut
          \begin{align*}
            n=1            & \implies \frac{1}{4}\left[\cancel{3\sin\left( \frac{x}{3} \right)} - \sin(x)\right]                                         \\
            n=2            & \implies \frac{1}{4}\left[\cancel{9\sin\left( \frac{x}{9} \right)} - \cancel{3\sin\left( \frac{x}{3} \right)}\right]        \\
            n=3            & \implies \frac{1}{4}\left[\cancel{27\sin\left( \frac{x}{27} \right)} - \cancel{9\sin\left( \frac{x}{9} \right)}\right]      \\
                           & \vdots                                                                                                                      \\
            n=k            & \implies \frac{1}{4}\left[3^k\sin\left( \frac{x}{3^k} \right) - \cancel{3^{k-1}\sin\left( \frac{x}{3^{k-1}} \right)}\right] \\
            \cline{1-2}
            \sum_{n=1}^{k} & 3^{n-1} \sin^3\left( \frac{x}{3^n} \right) = \frac{1}{4}\left[3^k\sin\left( \frac{x}{3^k} \right) - \sin(x)\right].
          \end{align*}
          Sehingga untuk $k\to\infty$, kita peroleh
          \[
            \lim_{k \to \infty} 3^k \sin\left( \frac{x}{3^k} \right) = \lim_{k \to \infty} x\cdot \frac{\sin\left( \frac{x}{3^k}\right)}{\frac{x}{3^k}} = x.
          \]
          Jadi bentuk sederhananya adalah
          \[
            \sum_{n=1}^{\infty} 3^{n-1} \sin^3\left( \frac{x}{3^n} \right) = \boxed{\frac{1}{4}\left[x - \sin(x)\right]}.
          \]
        \end{solution}
  \item Nilai
        \[
          \sup \left\{ \inf \left\{ 5(-1)^n - \left( \frac{m+1}{n} \right)^2 : n \geq m \right\} : m \in \mathbb{N} \right\}
        \]
        adalah \ldots
        \begin{solution}
          Misalkan \( S = \left\{ \inf \left\{ 5(-1)^n - \left( \frac{m+1}{n} \right)^2 : n \geq m \right\} : m \in \mathbb{N} \right\} \).

          Untuk setiap \( m \in \mathbb{N} \), kita akan mencari nilai infimum dari barisan himpunan \[ T_m = \left\{ 5(-1)^n - \left( \frac{m+1}{n} \right)^2 : n \geq m \right\}. \]

          Perhatikan bahwa untuk membuat ekspresi \( 5(-1)^n - \left( \frac{m+1}{n} \right)^2 \) menjadi kecil, kita perlu
          \begin{itemize}
            \item $5(-1)^n$ harus negatif, berarti \( n \) haruslah ganjil.
            \item \(\left( \frac{m+1}{n} \right)^2\) harus besar, berarti \( n \) haruslah sekecil mungkin.
          \end{itemize}
          Selanjutnya dengan sedikit melihat pola anggota himpunan \( T_m \) untuk setiap \( m\in \mathbb{N} \)
          \begin{align*}
            T_1    & = \left\{ 5(-1)^n - \left( \frac{2}{n} \right)^2 : n \geq 1 \right\} = \left\{ -9, 4, -\frac{49}{9},\ldots \right\} \implies \inf(T_1) = -9                          \\
            T_2    & = \left\{ 5(-1)^n - \left( \frac{3}{n} \right)^2 : n \geq 2 \right\} = \left\{ \frac{11}{4}, -6, \frac{71}{16}, \ldots \right\} \implies \inf(T_2) = -6              \\
            T_3    & = \left\{ 5(-1)^n - \left( \frac{4}{n} \right)^2 : n \geq 3 \right\} = \left\{ -\frac{61}{9}, 1, -\frac{121}{25}, \ldots \right\} \implies \inf(T_3) = -\frac{61}{9} \\
            T_4    & = \left\{ 5(-1)^n - \left( \frac{5}{n} \right)^2 : n \geq 4 \right\} = \left\{ \frac{55}{16}, -6, \frac{121}{25}, \ldots \right\} \implies \inf(T_4) = -6            \\
            \vdots & \quad\vdots \quad\vdots \quad\vdots
          \end{align*}
          Dari sedikit perhitungan diatas, kita mendapatkan sebuah pola dugaan yaitu
          \[\inf(T_m) = \begin{cases}
              -5-\left(1+\frac{1}{m}\right)^2,\quad & \text{untuk } m \text{ ganjil} \\
              -6,\quad                              & \text{untuk } m \text{ genap}
            \end{cases}\]
          Dengan demikian, kita dapat menyatakan \( S \) sebagai
          \[S = \left\{ -5-\left(1+\frac{1}{m}\right)^2 : m \text{ ganjil} \right\} \cup \left\{ -6 : m \text{ genap} \right\}.\]
          Untuk mencari supremum dari \( S \), kita perlu mencari nilai maksimum dari kedua himpunan tersebut. Karena untuk \( m \) genap hasilnya tetap -6, maka kita fokus pada himpunan pertama.
          Dengan menggunakan limit, kita dapatkan
          \[
            \lim_{m \to \infty} -5-\left(1+\frac{1}{m}\right)^2 = -6.
          \]
          Sehingga supremum dari \( S \) adalah $\boxed{-6}$.
        \end{solution}

  \item Diberikan fungsi kontinu \( f : [0,1] \to \mathbb{R} \) dengan \( |f(x)| \leq x \) untuk setiap \( x \in [0,1] \). \\
        Nilai terbesar yang mungkin dari
        \[
          \int_0^1 \left( (f(x))^2 - x^4 f(x) \right) \, dx
        \]
        adalah \ldots
        \begin{solution}
          Diketahui sifat fungsi \( f\) dan \(g \) yang terintegralkan pada interval \( [a,b] \) dan memenuhi \(f(x)\leq g(x)\) untuk setiap \(x\in[a,b]\), maka
          \[\int_a^b f(x) \, dx \leq \int_a^b g(x) \, dx.\]
          Dengan demikian, kita dapat mencari nilai maksimum dari integral di soal
          \begin{align*}
            \int_0^1 \left( (f(x))^2 - x^4 f(x) \right) \, dx & \leq \left|\int_{0}^{1}\left( (f(x))^2 - x^4 f(x) \right)\right| \\
                                                              & \leq \int_{0}^{1}\left|(f(x))^2- x^4 f(x)\right| \, dx
          \end{align*}
          Ketaksamaan segitiga mengatakan $\left|f^2(x)-x^4f(x)\right|\leq |f^2(x)|+|x^4f(x)|$ untuk setiap $x\in[0,1]$. Demikian juga dengan integralnya
          \begin{align*}
            \int_{0}^{1}\left|(f(x))^2- x^4 f(x)\right| \, dx & \leq \int_{0}^{1}\left|(f(x))^2\right| + \int_{0}^{1}\left|x^4 f(x)\right| \, dx \\
                                                              & \leq \int_{0}^{1}x^2 \, dx + \int_{0}^{1}x^5 \, dx                               \\
                                                              & = \left[ \frac{x^3}{3} + \frac{x^6}{6} \right]_{0}^{1}                           \\
                                                              & = \frac{1}{3} + \frac{1}{6} = \frac{1}{2}.
          \end{align*}
          Jadi, nilai terbesar yang mungkin dari integral tersebut adalah $\boxed{\frac{1}{2}}$.
        \end{solution}

  \item Banyaknya bilangan asli \( x \in \{1,2,3,\ldots,2025\} \) yang bukan kelipatan 2 dan bukan kelipatan 5 adalah \ldots
        \begin{solution}
          Misalkan \( S \) adalah himpunan bilangan asli dari 1 sampai 2025 dan. Selanjutnya definisikan $A_1$ dan $A_2$ sebagai himpunan bilangan asli dari 1 sampai 2025 yang masing-masing merupakan kelipatan 2 dan kelipatan 5. Dengan mudah kita peroleh bahwa
          \begin{flalign*}
            |S|            & = 2025,                                             & \\
            |A_1|          & = \left\lfloor \frac{2025}{2} \right\rfloor = 1012, & \\
            |A_2|          & = \left\lfloor \frac{2025}{5} \right\rfloor = 405,  & \\
            |A_1 \cap A_2| & = \left\lfloor \frac{2025}{10} \right\rfloor = 202.
          \end{flalign*}
          Menggunakan prinsip inklusi-eksklusi, maka banyaknya bilangan asli dari 1 sampai 2025 yang bukan kelipatan 2 dan bukan kelipatan 5 adalah
          \[|A_1^c \cap A_2^c| = |S| - |A_1| - |A_2| + |A_1 \cap A_2| = 2025 - 1012 - 405 + 202 = \boxed{810}.\]
        \end{solution}

  \item Tiga siswa $a_{1}$, $a_{2}$, $a_{3}$ dari Sekolah A dan 4 siswa $b_{1}$, $b_{2}$, $b_{3}$, $b_{4}$ dari Sekolah B berkumpul dalam sebuah pertemuan.
        Banyaknya cara menyusun ketujuh siswa tersebut dalam satu baris dengan syarat tidak terdapat satu blok yang berisikan semua siswa dari sekolah yang sama adalah \ldots
        (Contoh: $b_{3}b_{2}a_{1}a_{3}b_{1}b_{4}a_{2}$ diperbolehkan, tetapi $b_{1}b_{4}a_{1}a_{2}a_{3}b_{3}b_{2}$ dan $a_{1}b_{4}b_{3}b_{1}b_{2}a_{3}a_{2}$ tidak diperbolehkan)
        \begin{solution}
          Misalkan $S$ adalah himpunan semua susunan ketujuh siswa tersebut, sehingga jelas bahwa $|S|=7!=5040$. Misalkan $A$ dan $B$ adalah himpunan susunan masing-masing siswa sekolah A yang satu blok berurutan dan siswa sekolah B yang satu blok berurutan (Contoh: $b_1b_2a_{1}a_{2}a_{3}b_3b_4\in A$ dan $a_{3}b_1b_2b_3b_4a_{1}a_{2}\in B$). Banyaknya anggota $A,B$ dan $A\cap B$ dapat dihitung sebagai berikut:
          \begin{flalign*}
            |A|        & = 5! \cdot 3! = 120 \cdot 6 = 720,                & \\
            |B|        & = 4! \cdot 4! = 24 \cdot 24 = 576,                & \\
            |A \cap B| & = 2 \cdot 4! \cdot 3! = 2 \cdot 24 \cdot 6 = 288.
          \end{flalign*}
          Lagi-lagi dengan prinsip inklusi-eksklusi, maka banyaknya susunan ketujuh siswa tersebut yang tidak memiliki satu blok berisikan semua siswa dari sekolah yang sama adalah
          \[|A^c \cap B^c| = |S| - |A| - |B| + |A \cap B| = 5040 - 720 - 576 + 288 = 4.032\]
        \end{solution}

  \item Jika $S_{5}$ menyatakan grup semua fungsi bijektif pada $\{1,2,..., 5\}$ terhadap operasi komposisi fungsi, maka banyaknya elemen berorder 2 pada $S_{5}$ adalah \ldots % 
        \begin{solution}
          Misalkan $g\in S_5$ adalah elemen berorder 2, maka $g^2 = e$ dimana $e$ adalah elemen identitas pada $S_5$. Perhatikan bahwa untuk setiap elemen $g\in S_5$  haruslah berbentuk maksimal dua sikel yang saling asing dengan maksimal 2 elemen persikelnya yaitu
          \[
            g = \begin{pmatrix}
              a_1 & a_2
            \end{pmatrix}
            \text{ atau }g = \begin{pmatrix}
              a_1 & a_2
            \end{pmatrix}
            \begin{pmatrix}
              a_3 & a_4
            \end{pmatrix}
          \]
          sehingga banyaknya elemen berorder 2 pada $S_5$ adalah
          \[
            \binom{5}{2} + \frac{1}{2!}\binom{5}{2}\binom{3}{2}= 10 + 15 = \boxed{25}.
          \]
        \end{solution}
  \item Jika $z$, $p$, dan $m$ berturut-turut menyatakan banyaknya ideal di $\mathbb{Z}_{2025}$, banyaknya ideal prima di $\mathbb{Z}_{2025}$, dan banyaknya ideal maksimal di $\mathbb{Z}_{2025}$, maka nilai $z+p+m$ adalah \ldots % 
        \begin{solution}
          Perhatikan bahwa $2025 = 3^4 \cdot 5^2$. Dengan demikian, banyaknya ideal di $\mathbb{Z}_{2025}$ adalah
          \[
            z = (4+1)(2+1) = 15.
          \]
          Banyaknya ideal prima di $\mathbb{Z}_{2025}$ adalah
          \[
            p = 2,
          \]
          karena hanya ada dua bilangan prima yang membagi $2025$ yaitu $3$ dan $5$. Sedangkan banyaknya ideal maksimal di $\mathbb{Z}_{2025}$ adalah
          \[
            m = 2,
          \]
          karena ideal maksimal berkorespondensi satu-satu dengan bilangan prima yang membagi $2025$. Dengan demikian, kita peroleh
          \[
            z + p + m = 15 + 2 + 2 = \boxed{19}.
          \]
        \end{solution}
\end{enumerate}
\section*{Uraian}
\begin{enumerate}
  \item
        Diberikan ruang vektor $V$ atas lapangan $F$ dengan $\dim(V)=n$. Misalkan $U$ dan $W$ merupakan dua ruang bagian dari $V$ dengan $\dim(U)+\dim(W)=n$. % 
        Buktikan bahwa terdapat transformasi linear $T:V\longrightarrow V$ yang memenuhi $\ker(T)=U$ dan $\mathrm{Im}(T)=W.$ % 
        \begin{solution}
          Misalkan $\dim(U)=k$ dan $\dim(W)=n-k$. Karena $U$ dan $W$ adalah subruang dari $V$, maka terdapat basis $B_U=\{u_1,u_2,\ldots,u_k\}$ dari $U$ dan basis $B_W=\{w_1,w_2,\ldots,w_{n-k}\}$ dari $W$. Disisi lain dengan perluasan basis, kita dapatkan basis $B_V=\{u_1,u_2,\ldots,u_k,v_1,v_2,\ldots,v_{n-k}\}$ dari $V$.

          Selanjutnya perlu kita definisikan transformasi linear $T:V\longrightarrow V$ sebagai berikut
          \[
            T(u_i) = 0, \quad \text{untuk setiap } i\in\{1,2,\ldots,k\}
          \]
          dan
          \[
            T(v_j) = w_j, \quad \text{untuk setiap } j\in\{1,2,\ldots,n-k\}.
          \]
          Selanjutnya dengan definisi diatas akan dibuktikan dua hal berikut:
          \begin{itemize}
            \item $\ker(T)=U$ \\
                  Misalkan $x\in \ker(T)$, maka $T(x)=0$. Karena $B_V$ adalah basis dari $V$, maka $x$ dapat dituliskan sebagai
                  \[
                    x = a_1u_1 + a_2u_2 + \ldots + a_ku_k + b_1v_1 + b_2v_2 + \ldots + b_{n-k}v_{n-k}
                  \]
                  untuk beberapa skalar $a_i,b_j\in F$. Dengan menggunakan sifat linearitas dari $T$, kita peroleh
                  \begin{align*}
                    T(x) & = T(a_1u_1 + a_2u_2 + \ldots + a_ku_k + b_1v_1 + b_2v_2 + \ldots + b_{n-k}v_{n-k})    \\
                         & = a_1T(u_1) + a_2T(u_2) + \ldots + a_kT(u_k) + b_1T(v_1) + \ldots + b_{n-k}T(v_{n-k}) \\
                         & = 0 + 0 + \ldots + 0 + b_1w_1 + b_2w_2 + \ldots + b_{n-k}w_{n-k}                      \\
                         & = b_1w_1 + b_2w_2 + \ldots + b_{n-k}w_{n-k}.
                  \end{align*}
                  Karena $T(x)=0$, maka $b_j=0$ untuk setiap $j\in\{1,2,\ldots,n-k\}$ karena $B_W$ adalah basis dari $W$ (bebas linear). Dengan demikian, kita peroleh
                  \[
                    x = a_1u_1 + a_2u_2 + \ldots + a_ku_k,
                  \]
                  yang berarti bahwa $x\in U$. Jadi, $\ker(T)\subseteq U$.

                  Sebaliknya, misalkan $y\in U$. Maka terdapat skalar $c_i\in F$ sehingga
                  \[
                    y = c_1u_1 + c_2u_2 + \ldots + c_ku_k.
                  \]
                  sehingga diperoleh
                  \[
                    T(y) = c_1T(u_1) + c_2T(u_2) + \ldots + c_kT(u_k) = c_1\cdot 0 + c_2\cdot 0 + \ldots + c_k\cdot 0 = 0.
                  \]
                  Dengan demikian, $y\in \ker(T)$, sehingga $U\subseteq \ker(T)$. Jadi dapat disimpulkan $\ker(T)=U$.
            \item $\mathrm{Im}(T)=W$ \\
                  Misalkan $z\in \mathrm{Im}(T)$, maka terdapat $x\in V$ sehingga $T(x)=z$. Karena $B_V$ adalah basis dari $V$, maka $x$ dapat dituliskan sebagai
                  \[
                    x = a_1u_1 + a_2u_2 + \ldots + a_ku_k + b_1v_1 + b_2v_2 + \ldots + b_{n-k}v_{n-k}
                  \]
                  untuk beberapa skalar $a_i,b_j\in F$. Dengan menggunakan sifat linearitas dari $T$, kita peroleh
                  \begin{align*}
                    T(x) & = T(a_1u_1 + a_2u_2 + \ldots + a_ku_k + b_1v_1 + b_2v_2 + \ldots + b_{n-k}v_{n-k})    \\
                         & = a_1T(u_1) + a_2T(u_2) + \ldots + a_kT(u_k) + b_1T(v_1) + \ldots + b_{n-k}T(v_{n-k}) \\
                         & = 0 + 0 + \ldots + 0 + b_1w_1 + b_2w_2 + \ldots + b_{n-k}w_{n-k}                      \\
                         & = b_1w_1 + b_2w_2 + \ldots + b_{n-k}w_{n-k}.
                  \end{align*}
                  Artinya $z=b_1w_1 + b_2w_2 + \ldots + b_{n-k}w_{n-k}\in W$. Jadi, $\mathrm{Im}(T)\subseteq W$.

                  Sebaliknya, misalkan $w\in W$. Maka terdapat skalar $c_j\in F$ sehingga
                  \[
                    w = c_1w_1 + c_2w_2 + \ldots + c_{n-k}w_{n-k}.
                  \]
                  Perhatikan bahwa dapat kita definisikan $v=c_1v_1 + c_2v_2 + \ldots + c_{n-k}v_{n-k}\in V$. Sehingga diperoleh informasi
                  \begin{align*}
                    T(v) & = T(c_1v_1 + c_2v_2 + \ldots + c_{n-k}v_{n-k})       \\                                                  \\
                         & = c_1T(v_1) + c_2T(v_2) + \ldots + c_{n-k}T(v_{n-k}) \\
                         & = c_1w_1 + c_2w_2 + \ldots + c_{n-k}w_{n-k} = w.
                  \end{align*}
                  Karena kita dapat menemukan $v\in V$ sehingga $T(v)=w$, maka $w\in \mathrm{Im}(T)$, sehingga $W\subseteq \mathrm{Im}(T)$. Jadi dapat disimpulkan $\mathrm{Im}(T)=W$.
          \end{itemize}
        \end{solution}

  \item
        Diketahui $A_{1},A_{2},...,A_{n}$ merupakan titik-titik sudut sebuah poligon $n$ sisi beraturan yang termuat pada sebuah lingkaran dengan radius $r$ dan titik pusat $O(0,0)$. Jika $P$ merupakan titik di luar lingkaran yang terletak pada garis perpanjangan $OA_{1}$, buktikan bahwa % 
        \[
          \prod_{k=1}^{n}|PA_{k}|=|OP|^{n}-r^{n},
        \] % 
        dengan $|AB|$ menyatakan panjang ruang garis yang menghubungkan titik $A$ dan $B$. % 

        \begin{solution}

        \end{solution}

  \item
        Diberikan fungsi kontinu $f:[a,b]\rightarrow\mathbb{R}$ yang terdiferensial pada $(a, b)$ dengan $f(a)=f(b)$. % 
        Tunjukkan bahwa untuk setiap bilangan asli $n$, terdapat $n$ bilangan real berbeda, % 
        $c_{1},c_{2},...,c_{n}\in(a,b)$, yang memenuhi % 
        \[
          f^{\prime}(c_{1})+f^{\prime}(c_{2})+\cdot\cdot\cdot+f^{\prime}(c_{n})=0.
        \] % 

  \item
        Untuk suatu bilangan bulat tak negatif $n$ dan $m$, misalkan $D(m,n)$ menyatakan banyaknya solusi persamaan % 
        \[
          x_{1}+x_{2}+\cdot\cdot\cdot+x_{m}=n
        \] % 
        dengan $x_{i}\in\mathbb{N}$, untuk setiap $i\in\{1,2,...,m\}$ dan $x_{1}<x_{2}<\ldots<x_{m}$. Tunjukkan bahwa % 
        \[
          D(m,n)=D(m,n-m)+D(m-1,n-m).
        \] % 

  \item
        \begin{enumerate}
          \item Diketahui $A$, $B$, $C$ merupakan subgrup dari sebuah grup $G$. Jika $A\subseteq B$, $A\cap C=B\cap C$, dan $AC=BC$, buktikan bahwa $A=B$. % 
          \item Carilah contoh grup $G$ dan subgrup $A$, $B$, $C$ dari $G$ yang memenuhi $A\cap C=B\cap C$, dan $AC=BC$ tetapi $A\ne B$. % 
        \end{enumerate}
        Catatan: Jika $P$ dan $Q$ adalah subgrup dari $(G, *)$, maka $PQ$ didefinisikan sebagai % 
        \[
          PQ=\{p*q:p\in P,q\in Q\}.
        \] % 
\end{enumerate}
\end{document}