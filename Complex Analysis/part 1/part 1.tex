\documentclass[a4paper, 12pt]{article}

\usepackage{geometry}
\usepackage{tikz}
    \usetikzlibrary{patterns,decorations.pathreplacing,snakes,arrows.meta}
    \tikzset{box/.style={draw, thick, minimum width=1cm, minimum height=1cm}}
\usepackage{bigints}
\usepackage{tabularx}
\usepackage{fancyhdr}
\usepackage{amsfonts}
\usepackage{graphicx}
\usepackage{fontenc}
\usepackage{amssymb}
\usepackage{amsmath}
\usepackage{amsthm}
\usepackage{footmisc} 
\usepackage[colorlinks=true, linkcolor=red]{hyperref}
\usepackage{multicol}
\usepackage{enumitem}
\usepackage{cancel}
%\usepackage{wrapfig}
\usepackage{longtable}
\usepackage{array}
\usepackage{bm}
\usepackage{color}
\usepackage{textcomp}
\usepackage{xcolor}
\usepackage{mdframed}
\definecolor{darkcyan}{HTML}{0091A4}  
\definecolor{brightcyan}{HTML}{dcf0f2}
\newmdenv[  
topline=false,  
rightline=false,  
bottomline=false,  
leftline=true,  
linecolor=darkcyan,  
linewidth=3pt,  
backgroundcolor=brightcyan,  
frametitle={\textit{Solusi}:}
]{solution}  
\setlength{\multicolsep}{5.0pt plus 2.0pt minus 1.5pt}% 50% of original values
\geometry{a4paper, portrait, top=2.5cm, left=2.5cm, right=2.5cm, bottom=2.5cm}

\newcommand{\R}{\mathbb{R}}
\newcommand{\Z}{\mathbb{Z}}
\newcommand{\C}{\mathbb{C}}
\newcommand{\N}{\mathbb{N}}
\newcommand{\Q}{\mathbb{Q}}
\newcommand{\F}{\mathbb{F}}

\renewcommand{\baselinestretch}{1.2}

\fancyfoot[L]{\textit{5002221132}}
\fancyfoot[R]{\textit{Tetew}}
\renewcommand{\headrulewidth}{0pt}
\renewcommand{\footrulewidth}{2pt}
\pagestyle{fancy}
\pagenumbering{gobble}
\begin{document}

\begin{enumerate}
  \item Diberikan bilangan kompleks
        \[
          z = \cos\left(\frac{2\pi}{3}\right) + i \sin\left(\frac{2\pi}{3}\right)
        \]
        Tentukan nilai
        \[
          (1 + z)(1 + z^2)(1 + z^3) \cdots (1 + z^{2026})
        \]
        \begin{solution}
          Kita perhatikan bahwa $z^3 = 1$, sehingga diperoleh fakta bahwa $z-1=0$ atau $z^2+z+1=0$. Namun karna $z \neq 1$, maka jelas yang berlaku hanyalah $z^2 + z + 1 = 0$. Selanjutnya dapat ditinjau bahwa
          \[
            (1 + z)(1 + z^2) = 1 + z + z^2 + z^3 = 1 + z + z^2 + 1 = 2 + z + z^2 = 2 - 1 = 1.
          \]
          dan
          \[
            1+ z^3 = 1 + 1 = 2.
          \]
          Dengan demikian, kita memperoleh
          \[
            (1 + z)(1 + z^2)(1 + z^3) = 1 \cdot 2 = 2.
          \]

          Perlu kita ingat bahwa $z^3 = 1$ yang menunjukkan bahwa $z^{3k} = 1$ untuk setiap $k \in \mathbb{Z}$. Oleh karena itu, dengan meninjau bahwa $2026 = 3 \cdot 675 + 1$, kita memperoleh
          \begin{align*}
            (1 + z)(1 + z^2)(1 + z^3) \cdots (1 + z^{2026}) & = [(1 + z)(1 + z^2)(1 + z^3)]^{675}(1 + z)            \\
                                                            & = 2^{675}\left(\frac{1}{2}+\frac{\sqrt{3}}{2}i\right) \\
                                                            & = 2^{674}\left(1 + \sqrt{3}i\right).
          \end{align*}

        \end{solution}

  \item Misalkan $z_1, z_2, z_3 \in \mathbb{C}$ yang memenuhi $|z_1| = |z_2| = |z_3| = 2025$. Nilai dari
        \[
          \left| \frac{z_1z_2 + z_2z_3 + z_1z_3}{z_1 + z_2 + z_3} \right| = \cdots
        \]
        \begin{solution}
          Rumuskan persamaan yang diketahui sebagai berikut
          \[
            z_1\overline{z_1} = z_2\overline{z_2} = z_3\overline{z_3} = 2025^2.
          \]
          Selanjutnya dapat kita peroleh dua hubungan yaitu
          \[
            z_1=\frac{z_2\overline{z_2}}{\overline{z_1}} \quad \text{dan} \quad z_3=\frac{z_2\overline{z_2}}{\overline{z_3}}.
          \]
          Kemudian kita substitusi ke dalam ekspresi yang ingin kita cari, namun disini kita subtitusi untuk pembilangnya saja
          \begin{align*}
            \left| \frac{z_1z_2 + z_2z_3 + z_1z_3}{z_1 + z_2 + z_3} \right| & = \left| \frac{z_2\overline{z_2}z_2\left[ \frac{1}{\overline{z_1}} + \frac{1}{\overline{z_3}}+ \frac{\overline{z_2}}{\overline{z_3}\overline{z_1}} \right]}{z_1 + z_2 + z_3} \right| \\
                                                                            & = 2025^3 \left| \frac{\frac{\overline{z_3}+\overline{z_1}+\overline{z_2}}{\overline{z_1}\overline{z_3}}}{z_1 + z_2 + z_3} \right|                                                    \\
                                                                            & = 2025^3\frac{1}{|\overline{z_1}\overline{z_3}|} \left| \frac{\overline{z_3}+\overline{z_1}+\overline{z_2}}{z_1 + z_2 + z_3} \right|                                                 \\
                                                                            & = 2025^3\frac{1}{2025^2} \left| \frac{\overline{z_3+z_1+z_2}}{z_1 + z_2 + z_3} \right|                                                                                               \\
                                                                            & = 2025\cdot\frac{|z_3+z_1+z_2|}{|z_1 + z_2 + z_3|}                                                                                                                                   \\
                                                                            & = 2025\cdot 1 = 2025.
          \end{align*}
        \end{solution}

  \item Untuk $m,n \in \mathbb{C}$, tunjukkan bahwa
        \[
          |1 + m| + |1 + n| + |1 + mn| \geq 2
        \]

  \item Tunjukkan bahwa
        \[
          |z| \leq \frac{1}{3} \quad \text{jika dan hanya jika} \quad \left| \frac{6z - i}{2 + 3zi} \right| \leq 1
        \]
\end{enumerate}

\end{document}