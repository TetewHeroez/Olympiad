\documentclass[a4paper, 11pt]{article}
\usepackage{geometry}
\usepackage{tikz}
    \usetikzlibrary{patterns,decorations.pathreplacing,snakes,arrows.meta}
    \tikzset{box/.style={draw, thick, minimum width=1cm, minimum height=1cm}}
\usepackage{bigints}
\usepackage{tabularx}
\usepackage{fancyhdr}
\usepackage{amsfonts}
\usepackage{graphicx}
\usepackage{fontenc}
\usepackage{amssymb}
\usepackage{amsmath}
\usepackage{amsthm}
\usepackage{footmisc} 
\usepackage[colorlinks=true, linkcolor=red]{hyperref}
\usepackage{multicol}
\usepackage{enumitem}
\usepackage{cancel}
%\usepackage{wrapfig}
\usepackage{longtable}
\usepackage{array}
\usepackage{bm}
\usepackage{color}
\usepackage{textcomp}
\usepackage{xcolor}
\usepackage{mdframed}
\definecolor{darkcyan}{HTML}{0091A4}  
\definecolor{brightcyan}{HTML}{dcf0f2}
\newmdenv[  
topline=false,  
rightline=false,  
bottomline=false,  
leftline=true,  
linecolor=darkcyan,  
linewidth=3pt,  
backgroundcolor=brightcyan,  
frametitle={\textit{Solusi}:}
]{solution}  
\setlength{\multicolsep}{5.0pt plus 2.0pt minus 1.5pt}% 50% of original values
\geometry{a4paper, portrait, top=2.5cm, left=3cm, right=3cm, bottom=2.5cm}

\newcommand{\R}{\mathbb{R}}
\newcommand{\Z}{\mathbb{Z}}
\newcommand{\C}{\mathbb{C}}
\newcommand{\N}{\mathbb{N}}
\newcommand{\Q}{\mathbb{Q}}
\newcommand{\F}{\mathbb{F}}

\renewcommand{\baselinestretch}{1.2}

\fancyfoot[L]{\textit{5002221132}}
\fancyfoot[R]{\textit{Tetew}}
\renewcommand{\headrulewidth}{0pt}
\renewcommand{\footrulewidth}{2pt}
\usepackage{titlesec}
\titleformat{\section}
  {\centering\bfseries}   % format teks
  {}                            % nomor section
  {0pt}                         % jarak antara label dan judul
  {\MakeUppercase}              % ubah ke huruf besar semua

\titlespacing*{\section}
  {0pt}{2.5ex plus 2ex minus .5ex}{2ex plus .5ex} % jarak atas & bawah


\pagenumbering{gobble}
\begin{document}
\begin{center}
  \textbf{\large SIMULASI ONMIPA NASIONAL BIDANG MATEMATIKA}\\[1ex]
  Durasi pengerjaan: 240 menit
\end{center}

\section*{Subbidang Aljabar Linear}

Diberikan ruang hasil kali dalam berdimensi hingga $V$ atas lapangan $\mathbb{R}$.
Diketahui $\{ e_1, \ldots, e_n \}$ merupakan basis ortonormal untuk $V$ dan
$B = \{ v_1, \ldots, v_n \} \subseteq V$. Jika
\[
  \| e_i - v_i \| < \frac{1}{\sqrt{n}}
\]
untuk setiap $i \in \{ 1, \ldots, n \}$,
buktikan bahwa $B$ merupakan basis untuk $V$.
\vspace*{1cm}
\section*{Subbidang Analisis Real}

Diberikan fungsi $f : [0,2] \to (0, \infty)$ yang memenuhi $f''(x) \ge 0$ untuk setiap $x \in [0,2]$ dan
\[
  \int_{0}^{1} f(t)\, dt - \int_{1}^{2} \frac{dt}{f(t)} \le 1.
\]
Buktikan bahwa
\[
  \int_{0}^{2} f(t)\, dt \le 2f(2).
\]
\vspace*{1cm}
\section*{Subbidang Analisis Kompleks}

Diketahui $\displaystyle P(z) = \sum_{j=0}^{n} a_j z^j$ merupakan polinomial dengan derajat $n \ge 1$ dan semua akar $P(z) = 0$ termuat di dalam himpunan $\{ z \in \mathbb{C} : |z| < 1 \}$. Jika didefinisikan
\[
  P^*(z) = z^n \overline{P}\!\left( 1/z \right),
\]
dengan
$
  \displaystyle\overline{P}(z) = \sum_{j=0}^{n} \overline{a_{j}} z^j,
$
tunjukkan bahwa semua akar-akar dari persamaan
\[
  P(z) + P^*(z) = 0
\]
termuat di dalam himpunan $\{ z \in \mathbb{C} : |z| = 1 \}$.

\vspace*{1cm}


\section*{Subbidang Kombinatorika}

Tentukan banyaknya $N \subset \{ 1, 2, \ldots, 2025 \}$ dengan
$N = \{ a_1, a_2, \ldots, a_{45} \}$ dan
\[
  \sum_{i=1}^{45} a_i \equiv 0 \pmod{45}.
\]
\vspace*{1cm}
\section*{Subbidang Struktur Aljabar}

Diberikan himpunan tak kosong $G$ dan operasi biner
$\star : G \times G \to G$ yang memenuhi kondisi:

\begin{enumerate}
  \item Untuk setiap $a,b,c \in G$ berlaku $(a \star b) \star c = a \star (b \star c)$;
  \item Untuk setiap $a,b,c \in G$ berlaku jika $a \star c = b \star c$, maka $a = b$; dan
  \item Terdapat $u \in G$ sedemikian sehingga untuk setiap $a \in G$ berlaku $a \star a \star a = u \star a \star u$.
\end{enumerate}

Tunjukkan bahwa $(G, \star)$ merupakan grup komutatif.
\newpage
\pagestyle{fancy}
\begin{center}
  \textbf{\underline{SOLUSI}}
\end{center}
\end{document}