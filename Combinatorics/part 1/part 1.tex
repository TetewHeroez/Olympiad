\documentclass[a4paper, 12pt]{article}

\usepackage{geometry}
\usepackage{tikz}
    \usetikzlibrary{patterns,decorations.pathreplacing,snakes,arrows.meta}
    \tikzset{box/.style={draw, thick, minimum width=1cm, minimum height=1cm}}
\usepackage{bigints}
\usepackage{tabularx}
\usepackage{fancyhdr}
\usepackage{amsfonts}
\usepackage{graphicx}
\usepackage{fontenc}
\usepackage{amssymb}
\usepackage{amsmath}
\usepackage{amsthm}
\usepackage{footmisc} 
\usepackage[colorlinks=true, linkcolor=red]{hyperref}
\usepackage{multicol}
\usepackage{enumitem}
\usepackage{cancel}
%\usepackage{wrapfig}
\usepackage{longtable}
\usepackage{array}
\usepackage{bm}
\usepackage{color}
\usepackage{textcomp}
\usepackage{xcolor}
\usepackage{mdframed}
\definecolor{darkcyan}{HTML}{0091A4}  
\definecolor{brightcyan}{HTML}{dcf0f2}
\newmdenv[  
topline=false,  
rightline=false,  
bottomline=false,  
leftline=true,  
linecolor=darkcyan,  
linewidth=3pt,  
backgroundcolor=brightcyan,  
frametitle={\textit{Solusi}:}
]{solution}  
\setlength{\multicolsep}{5.0pt plus 2.0pt minus 1.5pt}% 50% of original values
\geometry{a4paper, portrait, top=2.5cm, left=2.5cm, right=2.5cm, bottom=2.5cm}

\newcommand{\R}{\mathbb{R}}
\newcommand{\Z}{\mathbb{Z}}
\newcommand{\C}{\mathbb{C}}
\newcommand{\N}{\mathbb{N}}
\newcommand{\Q}{\mathbb{Q}}
\newcommand{\F}{\mathbb{F}}

\renewcommand{\baselinestretch}{1.2}

\fancyfoot[L]{\textit{5002221132}}
\fancyfoot[R]{\textit{Tetew}}
\renewcommand{\headrulewidth}{0pt}
\renewcommand{\footrulewidth}{2pt}
\pagestyle{fancy}
\pagenumbering{gobble}
\begin{document}

\begin{enumerate}
  \item \textbf{Definisi T-grid:} Sebuah T-grid adalah matriks berukuran $3 \times 3$ yang memenuhi dua syarat berikut:
        \begin{enumerate}
          \item Tepat lima dari entri-entri dalam matriks bernilai $1$, dan empat entri lainnya bernilai $0$.
          \item Di antara delapan baris, kolom, dan diagonal panjang (dua diagonal panjang adalah $\{a_{13}, a_{22}, a_{31}\}$ dan $\{a_{11}, a_{22}, a_{33}\}$), tidak boleh lebih dari satu yang memiliki ketiga entri bernilai sama.
        \end{enumerate}
        Tentukan banyaknya T-grid yang berbeda yang memenuhi syarat di atas.

  \item Berikan argumen kombinatorik untuk membuktikan
        \[
          \sum_{k=0}^{n} k \binom{n}{k}^2 = n \binom{2n - 1}{n - 1}
        \]

  \item Banyak solusi non-negatif dari persamaan
        \[
          2x_1 + x_2 + 3x_3 + 5x_4 = 35
        \]
        \begin{solution}
          Kita dapat menggunakan mfungsi pembangkit untuk menyelesaikan masalah ini. Pertama, kita akan mencari fungsi pembangkit dari setiap variabel:
          \begin{align*}
            x_1 & : g_1(x) = \frac{1}{1 - x^2}, \\
            x_2 & : g_2(x) = \frac{1}{1 - x},   \\
            x_3 & : g_3(x) = \frac{1}{1 - x^3}, \\
            x_4 & : g_4(x) = \frac{1}{1 - x^5}.
          \end{align*}
          Maka, fungsi pembangkit total adalah
          \[
            G(x) = \frac{1}{(1 - x^2)(1 - x)(1 - x^3)(1 - x^5)}.
          \]
          Kita perlu mencari koefisien dari $x^{35}$ dalam ekspansi dari $G(x)$.
        \end{solution}

  \item Tunjukkan bahwa untuk setiap bilangan irasional $x$ dan bilangan bulat positif $n$, terdapat bilangan rasional $\frac{p}{q}$ dengan $1 \leq q \leq n$ sehingga
        \[
          \left| x - \frac{p}{q} \right| < \frac{1}{nq}
        \]
\end{enumerate}
\end{document}