\documentclass[a4paper, 12pt]{article}

\usepackage{geometry}
\usepackage{tikz}
    \usetikzlibrary{patterns,decorations.pathreplacing,snakes,arrows.meta}
    \tikzset{box/.style={draw, thick, minimum width=1cm, minimum height=1cm}}
\usepackage{bigints}
\usepackage{tabularx}
\usepackage{fancyhdr}
\usepackage{amsfonts}
\usepackage{graphicx}
\usepackage{fontenc}
\usepackage{amssymb}
\usepackage{amsmath}
\usepackage{amsthm}
\usepackage{footmisc} 
\usepackage[colorlinks=true, linkcolor=red]{hyperref}
\usepackage{multicol}
\usepackage{enumitem}
\usepackage{cancel}
%\usepackage{wrapfig}
\usepackage{longtable}
\usepackage{array}
\usepackage{bm}
\usepackage{color}
\usepackage{textcomp}
\usepackage{xcolor}
\usepackage{mdframed}
\definecolor{darkcyan}{HTML}{0091A4}  
\definecolor{brightcyan}{HTML}{dcf0f2}
\newmdenv[  
topline=false,  
rightline=false,  
bottomline=false,  
leftline=true,  
linecolor=darkcyan,  
linewidth=3pt,  
backgroundcolor=brightcyan,  
frametitle={\textit{Solusi}:}
]{solution}  
\setlength{\multicolsep}{5.0pt plus 2.0pt minus 1.5pt}% 50% of original values
\geometry{a4paper, portrait, top=2.5cm, left=2.5cm, right=2.5cm, bottom=2.5cm}

\newcommand{\R}{\mathbb{R}}
\newcommand{\Z}{\mathbb{Z}}
\newcommand{\C}{\mathbb{C}}
\newcommand{\N}{\mathbb{N}}
\newcommand{\Q}{\mathbb{Q}}
\newcommand{\F}{\mathbb{F}}

\renewcommand{\baselinestretch}{1.2}

\fancyfoot[L]{\textit{5002221132}}
\fancyfoot[R]{\textit{Tetew}}
\renewcommand{\headrulewidth}{0pt}
\renewcommand{\footrulewidth}{2pt}
\pagestyle{fancy}
\pagenumbering{gobble}
\begin{document}
\begin{enumerate}
  \item Misalkan $f \in C^2[0, N]$ dan $|f'(x)| < 1$, $f''(x) > 0$ untuk setiap $x \in [0, N]$. Misalkan $0 \le m_0 < m_1 < \cdots < m_k \le N$ adalah bilangan bulat sedemikian sehingga $n_i = f(m_i)$ juga merupakan bilangan bulat untuk $i = 0, 1, \ldots, k$. Misalkan $b_i = n_i - n_{i-1}$ dan $a_i = m_i - m_{i-1}$ untuk $i = 1, 2, \ldots, k$. Buktikan bahwa
        \[
          -1 < \frac{b_1}{a_1} < \frac{b_2}{a_2} < \cdots < \frac{b_k}{a_k} < 1.
        \]
        \begin{solution}
          Diketahui bahwa $f$ kontinu dan $f''(x) > 0$, sehingga $f$ adalah fungsi yang cekung ke atas. Dengan menggunakan teorema nilai rata-rata, kita dapat menyatakan bahwa terdapat $c_i \in (m_{i-1}, m_i)$ sehingga
          \[
            f'(c_i)= \frac{f(m_i) - f(m_{i-1})}{m_i - m_{i-1}} = \frac{b_i}{a_i}.
          \]
          untuk setiap $i = 1, 2, \ldots, k$. Dengan kata lain kita punya informasi bahwa $-1 < f'(c_i) < 1$ untuk setiap $i$.

          Jelas bahwa $c_i<c_{i+1}$ untuk setiap $i=1, 2, \ldots, k$. Sekarang gunakan fakta bahwa $f'(x)$ adalah fungsi kontinu yang monoton naik, sebab $f\in C^2[0, N]$ dan $f''(x) > 0$ untuk setiap $x \in [0, N]$. Oleh karena itu berlaku
          \[
            -1 < f'(c_1) < f'(c_2) < \cdots < f'(c_k) < 1.
          \]
          Dengan demikian, kita memperoleh
          \[
            -1 < \frac{b_1}{a_1} < \frac{b_2}{a_2} < \cdots < \frac{b_k}{a_k} < 1.
          \]
        \end{solution}

  \item Misalkan $F : (1, \infty) \rightarrow \mathbb{R}$ adalah fungsi yang didefinisikan oleh
        \[
          F(x) := \int_{x}^{x^2} \frac{dt}{\ln t}.
        \]
        Tunjukkan bahwa $F$ adalah fungsi satu-ke-satu (injektif) dan tentukan daerah hasil (range) dari $F$.

  \item Misalkan $x_1 = 0.8$ dan $y_1 = 0.6$. Didefinisikan
        \[
          x_{n+1} = x_n \cos y_n - y_n \sin y_n, \quad y_{n+1} = x_n \sin y_n + y_n \cos y_n, \quad \text{untuk semua } n \ge 1.
        \]
        Tentukan konvergensi dari barisan-barisan tersebut, dan jika konvergen, tentukan nilai limitnya.

  \item Buktikan bahwa jika $f : [0, 1] \rightarrow [0, 1]$ adalah fungsi kontinu, maka barisan iterasi
        \[
          x_{n+1} = f(x_n)
        \]
        konvergen jika dan hanya jika
        \[
          \lim_{n \to \infty} (x_{n+1} - x_n) = 0.
        \]

  \item Diketahui fungsi kontinu $f : [0, 1] \rightarrow \mathbb{R}$ memenuhi
        \[
          x f(y) + y f(x) \le 1, \quad \text{untuk semua } x, y \in [0, 1].
        \]
        Tunjukkan bahwa:
        \[
          \int_0^1 f(x) \, dx \le \frac{\pi}{4}.
        \]
\end{enumerate}
\end{document}