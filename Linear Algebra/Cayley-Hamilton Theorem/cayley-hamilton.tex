\documentclass[12pt, a4paper]{article}

%=============== PREAMBLE ===============
\usepackage[indonesian]{babel}
\usepackage[utf8]{inputenc}
\usepackage{amsmath, amssymb} % Paket matematika dasar
\usepackage[a4paper, margin=2.5cm]{geometry} % Mengatur margin halaman
\usepackage[svgnames]{xcolor} % Untuk definisi warna
\usepackage{hyperref} % Untuk hyperlink (opsional)

%--- TCOLORBOX SETUP ---
% Memuat paket tcolorbox dengan library yang diperlukan
% skins: untuk styling tingkat lanjut (warna, gradien, dll.)
% theorems: untuk membuat environment teorema otomatis
% breakable: agar box bisa terpotong antar halaman
\usepackage[many]{tcolorbox}
\tcbuselibrary{skins, theorems, breakable}

%--- Palet Warna ---
\definecolor{imc_blue}{HTML}{003F5C}
\definecolor{imc_lightblue}{HTML}{E8F1F2}
\definecolor{imc_orange}{HTML}{FFA600}
\definecolor{imc_grey}{HTML}{F0F0F0}
\definecolor{imc_red}{HTML}{D44D5C}

%--- Definisi Box Teorema (Lebih Kokoh) ---
% Menggunakan format title langsung untuk menghindari konflik karakter ':'
\newtcbtheorem{mytheorem}{Teorema}%
{
    enhanced,
    colback=imc_lightblue,
    colframe=imc_blue,
    fonttitle=\bfseries,
    title={#1~\thetcbcounter}, % Format judul yang lebih aman
    attach boxed title to top left={yshift=-2mm, xshift=3mm},
    boxed title style={sharp corners, colback=imc_blue},
    sharp corners,
    breakable
}{th}

\newtcbtheorem{mydef}{Definisi}%
{
    enhanced,
    colback=imc_grey,
    colframe=gray!75!black,
    fonttitle=\bfseries,
    title={#1~\thetcbcounter},
    attach boxed title to top left={yshift=-2mm, xshift=3mm},
    boxed title style={sharp corners, colback=gray!75!black},
    sharp corners,
    breakable
}{def}

\newtcbtheorem{myexample}{Contoh Soal}%
{
    enhanced,
    colback=LightGoldenrodYellow!20,
    colframe=imc_orange,
    fonttitle=\bfseries,
    title={#1~\thetcbcounter},
    attach boxed title to top left={yshift=-2mm, xshift=3mm},
    boxed title style={sharp corners, colback=imc_orange},
    sharp corners,
    breakable
}{ex}

%--- Definisi Box Bukti (Disederhanakan) ---
% Menggunakan 'title' standar, lebih aman daripada 'overlay'
\newtcolorbox{myproof}{
    enhanced,
    colframe=cyan!60!black,
    colback=cyan!10,
    boxrule=0.5pt,
    sharp corners,
    breakable,
    fonttitle=\bfseries\itshape,
    title=Bukti:
}

\newtcolorbox{mywarning}[1][Penting]{
    enhanced,
    colback=red!5!white,
    colframe=imc_red,
    boxrule=0.8pt,
    fonttitle=\bfseries,
    title=#1
}


%=============== DOCUMENT ===============
\begin{document}

\begin{mydef}{Polinomial Karakteristik}
  Diberikan sebuah matriks persegi $A$ berukuran $n \times n$ atas lapangan $\mathbb{F}$. Polinomial karakteristik dari $A$, dinotasikan $p_A(\lambda)$, didefinisikan sebagai:
  \[ p_A(\lambda) = \det(A - \lambda I) \]
  di mana $I$ adalah matriks identitas $n \times n$.
\end{mydef}

Dengan definisi ini, kita dapat menyatakan Teorema Cayley-Hamilton.

\begin{mytheorem}{Cayley-Hamilton}
  Setiap matriks persegi $A$ berukuran $n \times n$ memenuhi persamaan karakteristiknya sendiri. Artinya, jika $p_A(\lambda)$ adalah polinomial karakteristik dari $A$, maka:
  \[ p_A(A) = \mathbf{0} \]
  di mana $\mathbf{0}$ adalah matriks nol berukuran $n \times n$.
\end{mytheorem}

\begin{myproof}
  Misalkan $A$ adalah matriks $n \times n$ dan $p_A(\lambda) = \det(A-\lambda I)$ adalah polinomial karakteristiknya.
  Misalkan $B(\lambda) = \text{adj}(A - \lambda I)$ adalah matriks adjugat dari $(A-\lambda I)$.

  Setiap entri dari $B(\lambda)$ adalah kofaktor dari $(A-\lambda I)$, yang merupakan polinomial dalam $\lambda$ dengan derajat paling tinggi $n-1$. Oleh karena itu, kita dapat menulis $B(\lambda)$ sebagai sebuah polinomial dengan koefisien berupa matriks:
  \[ B(\lambda) = B_{n-1}\lambda^{n-1} + B_{n-2}\lambda^{n-2} + \dots + B_1\lambda + B_0 \]
  di mana $B_i$ adalah matriks-matriks $n \times n$.

  Dari sifat dasar matriks adjugat, kita tahu bahwa:
  \[ (A-\lambda I) \cdot \text{adj}(A - \lambda I) = \det(A - \lambda I) \cdot I \]
  Substitusikan ekspresi untuk $B(\lambda)$ dan $p_A(\lambda) = c_n\lambda^n + \dots + c_0$:
  \[ (A-\lambda I) (B_{n-1}\lambda^{n-1} + \dots + B_0) = (c_n\lambda^n + \dots + c_0) I \]
  Dengan menjabarkan dan menyamakan koefisien untuk setiap pangkat $\lambda$, kita peroleh sistem persamaan matriks berikut:
  \begin{align*}
    -B_{n-1}           & = c_n I     \\
    AB_{n-1} - B_{n-2} & = c_{n-1} I \\
                       & \vdots      \\
    AB_1 - B_0         & = c_1 I     \\
    AB_0               & = c_0 I
  \end{align*}
  Kalikan setiap persamaan di atas dari kiri dengan perpangkatan $A$ yang sesuai ($A^n, A^{n-1}, \dots, I$) lalu jumlahkan semuanya. Sisi kiri akan menjadi deret teleskopik yang hasilnya adalah matriks nol, sementara sisi kanan menjadi $p_A(A)$. Maka, kita sampai pada kesimpulan:
  \[ p_A(A) = c_n A^n + c_{n-1} A^{n-1} + \dots + c_1 A + c_0 I = \mathbf{0} \]
\end{myproof}

\begin{myexample}
Verifikasi Teorema Cayley-Hamilton untuk matriks $A = \begin{pmatrix} 3 & 1 \\ 2 & 4 \end{pmatrix}$.

\textbf{Solusi:}\\
\textbf{Langkah 1: Cari Polinomial Karakteristik}
\begin{align*}
  p_A(\lambda) & = \det(A - \lambda I) = \det \begin{pmatrix} 3-\lambda & 1 \\ 2 & 4-\lambda \end{pmatrix} \\
               & = (3-\lambda)(4-\lambda) - (1)(2) = \lambda^2 - 7\lambda + 10
\end{align*}
\textbf{Langkah 2: Substitusi $A$ ke dalam Polinomial} \\
Teorema Cayley-Hamilton memprediksi bahwa $A^2 - 7A + 10I = \mathbf{0}$. Mari kita buktikan.
\[ A^2 = \begin{pmatrix} 3 & 1 \\ 2 & 4 \end{pmatrix} \begin{pmatrix} 3 & 1 \\ 2 & 4 \end{pmatrix} = \begin{pmatrix} 11 & 7 \\ 14 & 18 \end{pmatrix} \]
Maka,
\begin{align*}
  A^2 - 7A + 10I & = \begin{pmatrix} 11 & 7 \\ 14 & 18 \end{pmatrix} - 7\begin{pmatrix} 3 & 1 \\ 2 & 4 \end{pmatrix} + 10\begin{pmatrix} 1 & 0 \\ 0 & 1 \end{pmatrix} \\
                 & = \begin{pmatrix} 11 - 21 + 10 & 7 - 7 + 0 \\ 14 - 14 + 0 & 18 - 28 + 10 \end{pmatrix} = \begin{pmatrix} 0 & 0 \\ 0 & 0 \end{pmatrix} = \mathbf{0}
\end{align*}
Verifikasi berhasil.
\end{myexample}

\begin{enumerate}
  \item Misalkan $A$ adalah matriks $n \times n$ sedemikian sehingga $A^k = \mathbf{0}$ untuk suatu bilangan asli $k$ (matriks nilpoten). Buktikan bahwa semua nilai eigen dari $A$ adalah 0, dan simpulkan bahwa $A^n = \mathbf{0}$.
\end{enumerate}


\end{document}