\documentclass{article}
\usepackage[utf8]{inputenc}

\usepackage{hyperref,ragged2e,amsmath,multicol,gensymb,setspace,
fancyhdr,amsfonts,tikz,pgfplots,nccmath,enumerate,verbatim}
\usepackage[a4paper, width=216mm, height=297mm, margin=2.5cm]{geometry}
\usepgfplotslibrary{polar,fillbetween}
\usepgflibrary{shapes.geometric}
\usepgfplotslibrary{external}
\usetikzlibrary{calc,patterns,arrows}
\newcommand\mylog[1]{\mathop{{}^{#1}\mathrm{log}}}
\pgfplotsset{compat=1.15}
\pgfplotsset{my style/.append style={axis x line=middle, axis y line=
middle, xlabel={$x$}, ylabel={$y$}, axis equal }}
\usepackage{etoolbox}
\newcommand{\zerodisplayskips}{%
  \setlength{\abovedisplayskip}{0pt}%
  \setlength{\belowdisplayskip}{0pt}%
  \setlength{\abovedisplayshortskip}{0pt}%
  \setlength{\belowdisplayshortskip}{0pt}}
\hypersetup{
    colorlinks=true,
    linkcolor=blue,
    filecolor=blue,      
    urlcolor=blue,
}
\setlength{\columnsep}{0.8cm}

\pagestyle{fancy}

\begin{document}
\begin{enumerate}
    \item Diketahui $R$ merupakan suatu ring dengan identitas perkalian $1_R$ dan $(x+y)^2=x^2+y^2$ untuk setiap $x,y\in R$. Apakah $R$ merupakan ring komutatif?\\
    Untuk setiap $x,y\in R$, kita punya $$(x+y)^2=x^2+xy+yx+y^2=x^2+y^2$$ dan didapat $xy+yx=0$ atau $xy=-yx$. Kemudian karena $1_R\in R$ didapat pula untuk setiap $a\in R$
    \begin{align*}
        (a+1)^2=a^2+1.a+a.1+1^2&=a^2+1^2\\
        a^2+2a+1&=a^2+1\\
        2a&=0\\
        a&=-a
    \end{align*}
    Karena $a=-a$ untuk setiap $a\in R$, didapat $xy=yx$
    $$$$
    \item Jika $a,b,c,d$ merupakan bilangan bulat sehingga $f(x)=x^4+ax^3+bx^2+cx+d$ mempunyai akar $\sqrt{3}-\sqrt{5}$, maka nilai $a+b+c+d$ adalah \dots
    \item Diketahui $\mathcal{R}=\mathbb{Z}_{1013}\times \mathbb{Z}_{1013}$ merupakan ring terhadap operasi penjumlahan dan perkalian berikut:
    \begin{align*}
        (a,b)+(c,d)=((a+c)\mod 1013, (b+d)\mod 1013),\\
        (a,b)\cdot (c,d) = (ac\mod 1013, (ad+bc+bd) \mod 1013),
    \end{align*}
    untuk setiap $(a,b),(c,d)\in \mathcal{R}$. Buktikan terdapat tepat sebanyak 2024 elemen taknol di $\mathcal{R}$ yang merupakan pembagi nol. 
    \item Diberikan $G$ suatu grup siklis dengan order 2024. Banyak elemen $G$ yang berorde ganjil adalah \dots
    \item Diketahui $\mathbb{R}[x]$ adalah ring polinom atas $\mathbb{R}$. Untuk setiap $p(x)\in\mathbb{R}[x]$, ideal di $\mathbb{R}[x]$ yang dibangun oleh $p(x)$ dinotasikan sebagai $\langle p(x)\rangle$.
    \begin{enumerate}
        \item Buktikan $\langle x-2024\rangle$ merupakan ideal maksimal di $\mathbb{R}[x]$.
        \item Tentukan bilangan asli $a$ terbesar sehingga $\langle x^2+ax+2024\rangle$ merupakan ideal maksimal di $\mathbb{R}[x]$.
    \end{enumerate}
\end{enumerate}
\end{document}