\documentclass{article}
\usepackage[utf8]{inputenc}
\usepackage{amssymb,amsmath,amsfonts}
\usepackage{ragged2e,multicol,setspace, fancyhdr,enumerate,enumitem,cancel}
\usepackage[a4paper, width=216mm, height=297mm, margin=2.5cm]{geometry}
\usepackage{etoolbox}
\newcommand{\zerodisplayskips}{%
  \setlength{\abovedisplayskip}{0pt}%
  \setlength{\belowdisplayskip}{0pt}%
  \setlength{\abovedisplayshortskip}{0pt}%
  \setlength{\belowdisplayshortskip}{0pt}}
\usepackage{hyperref}
\hypersetup{
    colorlinks=true,
    linkcolor=blue,
    filecolor=magenta,      
    urlcolor=cyan,
    pdftitle={Overleaf Example},
    pdfpagemode=FullScreen,
    }
\setlength{\columnsep}{0.8cm}

\newcommand{\R}{\mathbb{R}}
\newcommand{\Z}{\mathbb{Z}}
\newcommand{\N}{\mathbb{N}}
\newcommand{\Q}{\mathbb{Q}}
\newcommand{\C}{\mathbb{C}}
\newcommand{\fpb}{\text{fpb}}
\newcommand{\kpk}{\text{kpk}}

\begin{document}
\setstretch{1.1}
\pagenumbering{gobble}
\begin{enumerate}
    \item Misalkan $G$ adalah grup berhingga komutatif sedemikian sehingga untuk semua $a\in G, a\neq e$ didapat $a^2=e$. Jika $a_1,a_2,\dots,a_n$ adalah semua elemen dari $G$ yang berbeda, maka hitung $a_1a_2\cdots a_n$
    \item Tunjukkan bahwa semua elemen taknol di $\mathbb{Z}_p$ dengan $p$ bilangan prima membentuk suatu grup terhadap perkalian mod $p$
    \item Buktikan bahwa jika $p$ prima maka $(p-1)!+1$ habis dibagi oleh $p$.
    \item Tunjukkan bahwa jika $H$ dan $K$ adalah subgrup dari $G$, maka $H\cap K$ adalah subgrup dari $G$.
    \item Tunjukkan bahwa jika $G$ adalah suatu grup dan sebarang elemen $a,b\in G$, maka $|aba^{-1}|=|b|$ dan $|ab|=|ba|$.
    \item Diberikan $G$ adalah suatu grup dengan operasi $\ast$. Jika $(a\ast b)^2=a^2\ast b^2$ untuk setiap $a,b\in G$, tunjukkan bahwa $G$ komutatif.
    \item Tunjukkan bahwa jumlah semua anggota $\mathbb{Z}_n$ habis dibagi $n$ untuk setiap $n$ ganjil.
    \item \textbf{(ONMIPA Wilayah 2024)} Suatu grup berhingga $(G,\ast)$ berorde $n$ dikatakan rapi jika terdapat $n$ unsur berbeda $g_1,g_2,\dots,g_n$ dari $G$ sehingga $G=\{g_1,\ast g_2, g_2\ast g_3,\dots, g_{n-1}\ast g_n, g_n\ast g_1\}$.
    \begin{enumerate}
        \item Tunjukkan bahwa $(\mathbb{Z}_7,+)$ rapi.
        \item Buktikan bahwa untuk setiap $n$ genap $(\mathbb{Z}_n,+)$ tidak rapi.
    \end{enumerate}
    \item \textbf{(KNMIPA Wilayah 2021)} Misalkan $S$ adalah suatu himpunan yang memiliki dua operasi biner $\circ$ dan $\ast$. Diketahui bahwa masing-masing operasi mempunyai unsur identitas (yang tidak mesti sama) dan untuk setiap $a,b,c,d\in S$ berlaku 
    \begin{align*}
        (a\ast b)\circ (c\ast d) = (a\circ c)\ast (b\circ d)
    \end{align*}
    Haruskah operasi biner $\circ$ dan $\ast$ merupakan operasi yang sama?
    \item \textbf{(ONMIPA Nasional 2022)} Misalkan $a,b,c$ adalah unsur di grup $G$ sehingga $abc=e$ dengan $e$ adalah unsur identitas di $G$. Untuk masing-masing pernyataan berikut, buktikan jika benar atau berikan contoh penyangkal jika salah.
    \begin{enumerate}
        \item $bca=e$
        \item $bac=e$.
    \end{enumerate}
\end{enumerate}
\newpage
\fancyhead[c]{\textbf{SOLUSI}}
\fancyfoot[L]{\textit{5002221132}}
\fancyfoot[R]{\textit{Tetew}}
\renewcommand{\headrulewidth}{0pt}
\renewcommand{\footrulewidth}{2pt}
\pagestyle{fancy}
\begin{enumerate}
    \item Karena $G$ berhingga, maka setiap $a\in G$ berakibat $|a|$ habis membagi $|G|$. Dapat diambil kesimpulan bahwa $G$ berorde genap. 
    
    \item Misalkan $a\in \Z_p$ dengan $a\neq 0$. Jelas bahwa perkalian di $\Z_p$ adalah \textbf{tertutup} dan \textbf{asosiatif}. Lalu terdapat \textbf{elemen identitas} $e=1$ sehingga $a\cdot 1=1 \cdot a=a$. 
    
    Selanjutnya karena $p$ prima, maka $\fpb(a,p)=1$ atau dengan menggunakan \href{https://id.wikipedia.org/wiki/Identitas_Bézout}{Identitas Bézout} terdapat $x,y\in \Z$ sehingga
    \[ax+py=1\]
    Dengan menerapkan modulo $p$ pada kedua ruas, didapat 
    \begin{align*}
        ax+py&\equiv 1 \mod p\\
        ax&\equiv 1 \mod p
    \end{align*}
    Dapat dilihat bahwa $x$ adalah invers dari $a$ sehingga setiap $a\neq 0$ \textbf{memiliki invers}. 

    $\therefore \, (\Z_p\setminus \{0\},\cdot)$ adalah grup.

    \item Misalkan kita bekerja di ring $\Z_p$ dengan $p$ prima, maka ring tersebut adalah lapangan sehingga setiap elemen taknol memiliki invers. Pertama kita tinjau elemen apa saja yang invers nya adalah dirinya sendiri. Jika $a \in \Z$ dan $a=a^{-1}$, maka
    \begin{align*}
        a^2&\equiv 1 \mod p\\
        a^2-1&\equiv 0 \mod p\\
        (a-1)(a+1)&\equiv 0 \mod p
    \end{align*}
    Dari sini kita mendapatkan bahwa $a\equiv 1 \mod p$ atau $a\equiv -1 \mod p$. Jadi elemen yang inversnya adalah dirinya sendiri adalah $1$ dan $p-1$.
    
    Selanjutnya tinjau elemen $\{2,3,\dots,p-2\}$ yang dimana setiap elemen tersebut memiliki invers sehingga karena banyaknya elemennya genap maka untuk setiap elemen pastilah mempunyai pasangan inversnya. Jadi kita bisa membagi elemen tersebut menjadi pasangan-pasangan $(a,a^{-1})$ sehingga perkalian dari semua elemen tersebut adalah $1$.
    \begin{align*}
        (p-1)!&=(p-1)(p-2)\cdots 2\cdot 1\\
        &=(p-1)[(p-2)(p-3)\cdots 2]\cdot 1\\
        &=(p-1)[\underbrace{(1)\cdot (1)\cdots (1)}_{\text{disusun ulang}}]\\
        &\equiv -1 \mod p
    \end{align*}
    Jadi $(p-1)!+1\equiv 0 \mod p$ atau $(p-1)!+1$ habis dibagi oleh $p$.
    
    \item Misalkan $m,n\in H$ dan $m,n\in K$ sehingga $m,n\in H\cap K$. Karena $H$ dan $K$ masing-masing adalah subgrup dari $G$, maka berlaku $mn^{-1}\in H$ dan $mn^{-1}\in K$. Dengan demikian didapatkan $mn^{-1}\in H\cap K$.

    $\therefore\, H\cap K$ adalah subgrup dari $G$.

    \item Misalkan $a,b\in G$ dengan $G$ grup. 
    \begin{itemize}
        \item Andaikan $|ab|=n$ dan $|ba|=m$, maka
        \begin{flalign*}
            (ab)^n=e\iff\underbrace{(ab)(ab)\cdots (ab)}_{n}=e\iff a\underbrace{(ba)\cdots (ba)}_{n-1}b=e&\\
            \iff\underbrace{(ba)\cdots (ba)}_{n-1}=a^{-1}b^{-1}\iff\underbrace{(ba)\cdots (ba)}_{n}=e\iff(ba)^n=e
        \end{flalign*}
        Dari hasil di atas didapatkan $n=k_1m$ dengan $k_1=1,2,3,\ldots$.\\

        Selanjutnya dapat kita tinjau 
        \begin{flalign*}
            (ba)^m=e\iff\underbrace{(ba)(ba)\cdots (ba)}_{m}=e\iff b\underbrace{(ab)\cdots (ab)}_{m-1}a=e&\\
            \iff\underbrace{(ab)\cdots (ab)}_{m-1}=a^{-1}b^{-1}\iff\underbrace{(ab)\cdots (ab)}_{m}=e\iff(ab)^m=e
        \end{flalign*}
        Dari hasil di atas didapatkan $m=k_2n$ dengan $k_2=1,2,3,\ldots$.\\

        Alhasil kita dapatkan $n=k_1m\longrightarrow n=k_1(k_2n)\longrightarrow k_1=k_2=1$. Sehingga didapatkan kesimpulan $|ab|=|ba|$.
        \item Dengan cara yang sama, seperti yang dilakukan pada poin pertama.\\
        Andaikan $|aba^{-1}|=n$, maka
        \begin{flalign*}
            (aba^{-1})^n=e\iff ab(a^{-1}a)b(a^{-1}a)\cdots (a^{-1}a)ba^{-1}=e&\\
            \iff a\underbrace{bb\cdots b}_na^{-1}=e\iff a^{-1}a\underbrace{bb\cdots b}_na^{-1}a=a^{-1}a\iff b^n
        \end{flalign*}
        Dari hasil di atas didapatkan $n=k_1|b|$.\\

        Kemudian andaikan $|b|=m$, maka dengan cara yang sama didapatkan $m=k_2|aba^{-1}|$. Disini nantinya berakibat $k_1=k_2=1$.
    \end{itemize}
    $\therefore \,|aba^{-1}|=|b|$.

    \item Dengan mudah kita bisa manipulasi persamaan yang telah Diketahui
    \begin{align*}
        (a\ast b)^2&=a^2\ast b^2\\
        a\ast b\ast a\ast b&=a\ast a\ast b\ast b\\
        \cancel{a}\ast b\ast a\ast \cancel{b}&=\cancel{a}\ast a\ast b\ast \cancel{b} \quad \text{(Kanselasi)}\\
        b\ast a&=a\ast b
    \end{align*}
    Sehingga $G$ komutatif.

    \item Anggota $\mathbb{Z}_n$ adalah $\{0,1,2,\dots,n-1\}$. Kemudian jumlah semua anggotanya adalah 
    \[0+1+2+\dots+n-1=\dfrac{n(n-1)}{2}\] 
    Selanjutnya karena $n$ ganjil berakibat $n-1$ genap, sehingga $\dfrac{n-1}{2}$ adalah bilangan bulat. Jadi terbukti $\dfrac{n(n-1)}{2}$ habis dibagi $n$ untuk $n$ ganjil.

    \item \begin{enumerate}
        \item Kita pilih unsur terurut berbeda dari $\mathbb{Z}_7$ yaitu $\{0,1,2,3,4,5,6\}$. Selanjutnya kita operasikan unsur berurutan tersebut
        \begin{multicols}{4}
            \begin{itemize}
                \item $0+1=1$
                \item $1+2=3$
                \item $2+3=5$
                \item $3+4=0$
                \item $4+5=2$
                \item $5+6=4$
                \item $6+0=6$
            \end{itemize}
        \end{multicols}
        Dapat dilihat bahwa hasil operasinya memmbentuk himpunan grup $\mathbb{Z}_7$.

        $\therefore\, (\mathbb{Z}_7,+)$ rapi.
        
        \item Tinjau jumlahan semua elemen dari $\mathbb{Z}_n$ dengan $n$ genap yang berurutan, yaitu
        \[0+1+2+\dots+n-1=\dfrac{n(n-1)}{2}\]
        Jika diasumsikan $\left.\dfrac{n(n-1)}{2}\,\right|\,n$, maka dapat dilihat bahwa $n-1\,|\,n$ atau $\left.\dfrac{n}{2}\,\right|\,n$. Namun kedua hal tersebut adalah pernyataan yang salah. Sehingga haruslah
        \[\dfrac{n(n-1)}{2}\nmid n\iff \dfrac{n(n-1)}{2}\not\equiv 0 \mod n\]
        Artinya jumlah semua elemen dari $\mathbb{Z}_n$ tidak habis dibagi $n$ untuk $n$ genap.\\

        Selanjutnya asumsikan bahwa $(\mathbb{Z}_n,+)$ rapi, maka berakibat elemen $\Z_n$ dapat dibentuk menjadi $\{g_1+g_2,g_2+g_3,\dots,g_{n-1}+g_n,g_n+g_1\}$. Kemudian jumlahan semua elemen tersebut adalah
        \[(g_1+g_2)+(g_2+g_3)+\dots+(g_{n-1}+g_n)+(g_n+g_1)=2\sum_{i=1}^{n}g_i=2\sum_{0}^{n-1}i=2\left(\dfrac{n(n-1)}{2}\right)=n(n-1)\]
        Sehingga haruslah $n(n-1)\equiv 0 \mod n$ atau jumlahan elemen $\Z_n$ habis dibagi $n$ untuk $n$ genap. Hal ini kontradiksi dengan pembuktian sebelumnya. \\

        $\therefore\, (\mathbb{Z}_n,+)$ tidak rapi.
    \end{enumerate}
    \item Karena masing-masing operasi mempunyai unsur identitas, dapat dimisalkan $e_\circ$ dan $e_\ast$ sebagai unsur identitas dari operasi $\circ$ dan $\ast$. sehingga untuk $x,y\in S$ berlaku
    \begin{align*}
        (x\ast e_\ast)\circ (y\ast e_\circ)&=(x\circ y)\ast (e_\ast\circ e_\circ)\\
        x\circ (y\ast e_\circ)&=(x\circ y)\ast e_\ast\\
        x\circ (y\ast e_\circ)&=x\circ y 
    \end{align*}
    subtitusi $x=e_\circ$ dan $y=e_\ast$ didapatkan 
    \begin{align*}
        e_\circ\circ(e_\ast\ast e_\circ)&=e_\circ\circ e_\ast\\
        e_\circ\circ e_\circ&=e_\ast\\
        e_\circ&=e_\ast
    \end{align*}
    Artinya unsur identitas dari operasi $\circ$ dan $\ast$ adalah sama, misalkan saja $e$. Selanjutnya dapat kita tinjau
    \begin{align*}
        (x\ast e)\circ (e\ast y)&=(x\circ e)\ast (e\circ y)\\
        x\circ y&=x\ast y
    \end{align*}
    Sehingga operasi biner $\circ$ dan $\ast$ haruslah sama.
    \item \begin{enumerate}
        \item Karena $a,b,c\in G$ grup dan $abc=e$, maka 
        \[a^{-1}a(bc)=a^{-1}e\iff bc=a^{-1}\iff bca=a^{-1}a\iff bca=e\,\blacksquare\]
        \item Kita ambil contoh grup yang tak komutatif yaitu $S_3$ dan $a=\begin{pmatrix}1&3\end{pmatrix},b=\begin{pmatrix}2&3\end{pmatrix},$ dan $c=\begin{pmatrix}1&2&3\end{pmatrix}$. Dengan mudah dapat dicek
        \begin{align*}
            abc&=\begin{pmatrix}1&3\end{pmatrix}\begin{pmatrix}2&3\end{pmatrix}\begin{pmatrix}1&2&3\end{pmatrix}=\begin{pmatrix}1&3\end{pmatrix}\begin{pmatrix}1&3\end{pmatrix}=(1)=e
        \end{align*}
        Namun
        \begin{align*}
            bac&=\begin{pmatrix}2&3\end{pmatrix}\begin{pmatrix}1&3\end{pmatrix}\begin{pmatrix}1&2&3\end{pmatrix}=\begin{pmatrix}2&3\end{pmatrix}\begin{pmatrix}1&2\end{pmatrix}=\begin{pmatrix}1&3&2\end{pmatrix}\ne e
        \end{align*}
        Sehingga pernyataan $bac=e$ salah.
    \end{enumerate}
\end{enumerate}
\end{document}
