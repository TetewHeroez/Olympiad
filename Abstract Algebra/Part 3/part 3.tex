\documentclass[a4paper, 12pt]{article}

\usepackage{geometry}
\usepackage{tikz}
    \usetikzlibrary{patterns,decorations.pathreplacing,snakes,arrows.meta}
    \tikzset{box/.style={draw, thick, minimum width=1cm, minimum height=1cm}}
\usepackage{bigints}
\usepackage{tabularx}
\usepackage{fancyhdr}
\usepackage{amsfonts}
\usepackage{graphicx}
\usepackage{fontenc}
\usepackage{amssymb}
\usepackage{amsmath}
\usepackage{amsthm}
\usepackage{footmisc} 
\usepackage[colorlinks=true, linkcolor=red]{hyperref}
\usepackage{multicol}
\usepackage{enumitem}
\usepackage{cancel}
%\usepackage{wrapfig}
\usepackage{longtable}
\usepackage{array}
\usepackage{bm}
\usepackage{color}
\usepackage{textcomp}
\usepackage{xcolor}
\usepackage{mdframed}
\definecolor{darkcyan}{HTML}{0091A4}  
\definecolor{brightcyan}{HTML}{dcf0f2}
\newmdenv[  
topline=false,  
rightline=false,  
bottomline=false,  
leftline=true,  
linecolor=darkcyan,  
linewidth=3pt,  
backgroundcolor=brightcyan,  
frametitle={\textit{Solusi}:}
]{solution}  
\setlength{\multicolsep}{5.0pt plus 2.0pt minus 1.5pt}% 50% of original values
\geometry{a4paper, portrait, top=2.5cm, left=2.5cm, right=2.5cm, bottom=2.5cm}

\newcommand{\R}{\mathbb{R}}
\newcommand{\Z}{\mathbb{Z}}
\newcommand{\C}{\mathbb{C}}
\newcommand{\N}{\mathbb{N}}
\newcommand{\Q}{\mathbb{Q}}
\newcommand{\F}{\mathbb{F}}

\renewcommand{\baselinestretch}{1.2}

\fancyfoot[L]{\textit{5002221132}}
\fancyfoot[R]{\textit{Tetew}}
\renewcommand{\headrulewidth}{0pt}
\renewcommand{\footrulewidth}{2pt}
\pagestyle{fancy}
\pagenumbering{gobble}
\begin{document}
\begin{enumerate}
  \item Diberikan $G$ adalah grup komutatif berorde $n$. Jika $p$ dan $q$ adalah dua bilangan prima yang berbeda dan keduanya membagi $n$, maka tunjukkan bahwa $G$ memuat suatu subgrup siklik berorde $pq$.

  \item Diberikan $R$ adalah ring komutatif dengan elemen identitas $1_R \in R$. Tunjukkan bahwa $R$ adalah lapangan jika dan hanya jika setiap ideal sejati di $R$ adalah ideal prima.

  \item Diberikan suatu grup $G$ sedemikian sehingga anggota dari $G$ yang berorde $7$ ada sebanyak $42$. Jika banyaknya subgrup dari $G$ yang berorde $7$ adalah $n$, maka $|\mathrm{Aut}(\mathbb{Z}_n)| = \cdots$.

  \item Misalkan $G$ adalah suatu subgrup dari $GL_2(\mathbb{R})$ yang dibangun oleh $A$ dan $B$ dengan
        \[
          A = \begin{pmatrix} 2 & 0 \\ 0 & 1 \end{pmatrix}, \quad
          B = \begin{pmatrix} 1 & 1 \\ 0 & 1 \end{pmatrix}.
        \]
        Misalkan $H \subset G$ berisi matriks-matriks
        \[
          \begin{pmatrix} a_{11} & a_{12} \\ a_{21} & a_{22} \end{pmatrix}
          \quad \text{dengan } a_{11} = a_{22} = 1.
        \]
        Tunjukkan bahwa $H$ adalah subgrup komutatif dari $G$.

  \item Diberikan ring $R$ yang tidak memiliki identitas perkalian $1$ dengan kuadrat dari tiap anggotanya adalah $0$. Buktikan bahwa $abc + abc = 0$ untuk setiap $a, b, c \in R$.

  \item Tunjukkan bahwa persamaan
        \[
          x_1^2 + x_2^2 + \cdots + x_k^2 = -1
        \]
        tidak memiliki solusi di lapangan $\mathbb{Q}(\sqrt[5]{3}e^{2\pi i/5})$.
\end{enumerate}

\end{document}